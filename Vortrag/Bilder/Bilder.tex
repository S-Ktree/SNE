\documentclass[german,hyperref={pdfpagelabels=false,breaklinks=true}]{beamer}
\usefonttheme[onlymath]{serif}
\usepackage{etex}
\usepackage[ngerman]{babel}
\usepackage[ansinew]{inputenc}
\usepackage{amsmath,amssymb,amstext,amsbsy} % American-Maths-Standardized-Symbols
%\usepackage{trfsigns} % Laplace Zeichen
%\usepackage{exscale}  % Richte Skalierung von Integral- und Summenzeichen
\usepackage{graphicx} % Einbinden von Grafiken
\usepackage{multimedia}
\usepackage[arrow, matrix, curve]{xy}
\usepackage{pifont}
%\usepackage{arydshln}
\usepackage{tikz}
\usepackage{pgfplots}
\usepackage{calc}
\usepackage{xparse}
\usepackage{siunitx}
\usepackage[european,smartlabels]{circuitikz}
\usetikzlibrary{intersections,decorations.pathreplacing,fit,calc,positioning,plothandlers,plotmarks,patterns}


\newlength{\hatchspread}
\newlength{\hatchthickness}
\newlength{\hatchshift}
\newcommand{\hatchcolor}{}
%\pgfplotsset{compat=newest}
% declaring the keys in tikz
\tikzset{hatchspread/.code={\setlength{\hatchspread}{#1}},
         hatchthickness/.code={\setlength{\hatchthickness}{#1}},
         hatchshift/.code={\setlength{\hatchshift}{#1}},% must be >= 0
         hatchcolor/.code={\renewcommand{\hatchcolor}{#1}}}
% setting the default values
\tikzset{hatchspread=3pt,
         hatchthickness=0.4pt,
         hatchshift=0pt,% must be >= 0
         hatchcolor=black}
% declaring the pattern
\pgfdeclarepatternformonly[\hatchspread,\hatchthickness,\hatchshift,\hatchcolor]% variables
   {custom north west lines}% name
   {\pgfqpoint{\dimexpr-2\hatchthickness}{\dimexpr-2\hatchthickness}}% lower left corner
   {\pgfqpoint{\dimexpr\hatchspread+2\hatchthickness}{\dimexpr\hatchspread+2\hatchthickness}}% upper right corner
   {\pgfqpoint{\dimexpr\hatchspread}{\dimexpr\hatchspread}}% tile size
   {% shape description
    \pgfsetlinewidth{\hatchthickness}
    \pgfpathmoveto{\pgfqpoint{0pt}{\dimexpr\hatchspread+\hatchshift}}
    \pgfpathlineto{\pgfqpoint{\dimexpr\hatchspread+0.15pt+\hatchshift}{-0.15pt}}
    \ifdim \hatchshift > 0pt
      \pgfpathmoveto{\pgfqpoint{0pt}{\hatchshift}}
      \pgfpathlineto{\pgfqpoint{\dimexpr0.15pt+\hatchshift}{-0.15pt}}
    \fi
    \pgfsetstrokecolor{\hatchcolor}
%    \pgfsetdash{{1pt}{1pt}}{0pt}% dashing cannot work correctly in all situation this way
    \pgfusepath{stroke}
   }


\usetheme{default} %damit keine Kopfzeile, die viel Platz wegnimmt.
\usecolortheme[RGB={20, 30,100}]{structure}  %leichtes Blau, �hnlich dem FAU Logo
\useinnertheme{circles}  % Damit auch Gliederung sch�n nummeriert wird

\setbeamercovered{transparent}
\beamertemplatenavigationsymbolsempty %damit keine Navigationssymbole
\setbeamerfont{frametitle}{size=\large}
\setbeamersize{text margin left=10mm}
\setbeamersize{text margin right=5mm}
\newcommand{\dd}{\cdot}
\renewcommand{\tt}{(t)}
%\newcommand{\bs}{\boldsymbol}
\renewcommand{\d}{\mathrm{d}}
\renewcommand{\vec}[1]{\bs{\mathrm{#1}}}
\newcommand{\up}[1]{\mathrm{#1}}
\newcommand{\x}{\up{x}}
\renewcommand{\u}{\up{u}}
\newcommand{\C}{\up{C}}
\newcommand{\R}{\up{R}}
\newcommand{\T}{\up{T}}
\newcommand{\f}{\up{f}}
\newcommand{\eh}{\frac{1}{2}}
\renewcommand{\c}{\up{c}}
\renewcommand{\b}{\up{b}}
\renewcommand{\a}{\up{a}}
\newcommand{\s}{\up{s}}
\newcommand{\ul}[1]{\underline{#1}}
\newcommand{\com}[1]{\underline{\up{#1}}}
\newcommand{\comv}[1]{\underline{#1}}

\newcommand{\adj}{\mathrm{adj}}
%%%%%%%%%%%%%%%%%%%%%%%%%%
%TIKZ:
\tikzset{%
	>=latex,
  highlight/.style={rectangle,rounded corners,draw=orange,very thin,fill opacity=0.5,inner sep=-2pt},
  frame/.style={
	rectangle,minimum size=9mm,thick,draw=black
	},
 sum/.style={
	circle,minimum size =3mm,thick,draw=black, inner sep = 0pt
	},
 knot/.style={
	circle,minimum size =2mm,fill=black, inner sep=0pt
	},
 skip  loop/.style={to  path={--  ++(0,#1)  -|  (\tikztotarget)}},
 nonlin/.style = {frame, double,double distance=1pt},
 times/.style = {alias=sourcenode,frame, 
        append after command={
        ($(sourcenode.center)+(-0.2,0.2)$)--($(sourcenode.center)+(0.2,-0.2)$)
	  ($(sourcenode.center)+(-0.2,-0.2)$)--($(sourcenode.center)+(0.2,0.2)$)	
        }
	}
}



\newcommand{\tikzrightbrace}[4][0.5\textwidth]{%
%%% Ein rechter brace der die nodes 2 und 3 umschlie�t.
%%% rechts daneben wird der Text 4 mit der Breite 1 *optional angezeigt!
\tikz[overlay,remember picture]{
\node[fit=(#2.north east) (#3.south east),inner sep =0] (gesamt#2) {};
\draw[decorate, decoration=brace] (gesamt#2.north east)--(gesamt#2.south east);
\path (gesamt#2.east) node[right,xshift=2mm]{\parbox{#1}{#4}};
}
}


\newcommand{\tikzmark}[3][3pt]{\tikz[remember picture,baseline=(#2.base)] \node[inner sep=#1] (#2) {#3};}
%tikzmark: markiert inhalt als tikznode: 1(optional): inner sep der node
%									2: name der Node
%									3: Inhalt der Node
\newcommand{\tikzmarktwo}[3][2pt]{\tikz[remember picture,baseline=(#2.base)] 
\node[inner ysep=#1,inner xsep=0pt, outer xsep=0pt] (#2) {#3};}
%%tikzmark: markiert inhalt als tikznode: 1(optional): inner ysep der node
%%									2(optional): inner xsep der node
%%									3: name der Node
%%									4: Inhalt der Node

%% Befehl, um einfach Beschriftungen mit Pfeil hinzuzuf�gen: (h�lt Abstand nach unten/oben automatisch ein! (standardm��ig in footnotesize aber durch voranstellen einer gr��e �berschreibbar!)
% #1 (optional, def 0.5cm): Abstand der Beschriftung vom zu Beschriftenden
% #2 (optional, def 270): Winkel, in dem die Beschriftung angebracht wird.
% #3 zu beschriftender Text
% #4 Beschriftung 
\DeclareDocumentCommand \pin {O{0.5cm} O{270} m m}{
\tikz[baseline=(mynode.base), pin distance=#1,every pin edge/.style={<-}, inner sep=1pt]{
\node(mynode)[pin=#2:{\footnotesize  #4}]{#3};
\coordinate (pin) at (current bounding box.south);
\path let \p1=(pin), \p2=(mynode.center) in %
	node[inner sep=0pt](da) at (\x2,\y1){} node(gesamt)[fit=(mynode)(da),inner sep =0pt]{};
\pgfresetboundingbox
\useasboundingbox (gesamt.north west)rectangle(gesamt.south east);
%\draw[fill=red] (current bounding box.south west) circle(0.02cm);
%\draw[fill=red] (current bounding box.south east) circle(0.02cm);
}
}

\DeclareDocumentCommand \pintest{O{0.5cm} O{270} m m}{#1 #2 #3 #4}

\newcommand{\Highlight}[1][submatrix]{%
    \tikz[overlay,remember picture]{
    \node[highlight,fit=(left.north west) (right.south east)] (#1) {};}
}
\newcommand{\highlight}[2]{%
	\tikz[overlay,remember picture]{
	\node[highlight,fit=(#1.north west)(#1.south east)] (#2){};
	}
}
\newcommand{\underbrac}[2]{%
	\tikz[overlay,remember picture, very thin, orange]{
	\node[inner sep=-2pt,fit=(#1.north west)(#1.south east)] (#2){};
	\path (#1.south west)++(0.065,0.05) edge($(#1.south west)+(0.065,0)$)
		  ($(#1.south west)+(0.065,0)$) edge ($(#1.south east)-(0.065,0)$)
  	 	 ($(#1.south east)-(0.065,0)$)++(0,0.05) edge  ($(#1.south east)-(0.065,0)$);
	}
}
%%%%%%%%%%%%%%%%%%%%%%%%%%%%%%%%%%%%%%%
\newcommand{\tikzdoubleul}[1]{ % doppelte Unterstreichungen auch fuer groessere Objekte
\tikzmarktwo{tikzdoubleul}{#1}
\tikz[overlay, remember picture]{
\path (tikzdoubleul.south west) edge[double, thick, double distance = 1.5pt] (tikzdoubleul.south east);
}
}

\newcommand{\uldash}[1]{ % gestrichelte Unterstreichung
\tikzmarktwo[2pt]{uldash}{#1}
\tikz[overlay, remember picture]{
\path (uldash.south west) edge[draw, thick, dashed] (uldash.south east);
}
}

\newcommand{\tikzcancel}[1]{ % cancel gestrichelt
\tikzmarktwo{tikzcancel}{#1}
\tikz[overlay, remember picture]{
\path ($(tikzcancel.south west)-(1pt,2pt)$) edge[draw,dotted,thick,color =gray] ($(tikzcancel.north east)+(1pt,2pt)$);
}
}


%%%%%%%%%%%%%%%%%%%%%%%%%%%%%%
% fuer bessere Underbraces!!
% normal den underbrace Befehl verwenden aber dann den Text mathclapen! Bsp: \underbrace{a}{\mathclap{text}}
 \def\mathllap{\mathpalette\mathllapinternal}
 \def\mathllapinternal#1#2{%
 \llap{$\mathsurround=0pt#1{#2}$}%  $
 }
 \def\clap#1{\hbox  to  0pt{\hss#1\hss}}
 \def\mathclap{\mathpalette\mathclapinternal}
 \def\mathclapinternal#1#2{%
 \clap{$\mathsurround=0pt#1{#2}$}%
}
 \def\mathrlap{\mathpalette\mathrlapinternal}
 \def\mathrlapinternal#1#2{%
 \rlap{$\mathsurround=0pt#1{#2}$}%  $
 }
%%%%%%%%%%%%%%%%%%%%%%%%%%%%%%
%%%%%%%%%%%%%%%%%%%%%%%%%%%%
% underbrackets!
\makeatletter
\def\overbracket#1{\mathop{\vbox{\ialign{##\crcr\noalign{\kern3\p@}
\downbracketfill\crcr\noalign{\kern3\p@\nointerlineskip}
$\hfil\displaystyle{#1}\hfil$\crcr}}}\limits}
\def\underbracket#1{\mathop{\vtop{\ialign{##\crcr
$\hfil\displaystyle{#1}\hfil$\crcr\noalign{\kern3\p@\nointerlineskip}
\upbracketfill\crcr\noalign{\kern3\p@}}}}\limits}
\def\overparenthesis#1{\mathop{\vbox{\ialign{##\crcr\noalign{\kern3\p@}
\downparenthfill\crcr\noalign{\kern3\p@\nointerlineskip}
$\hfil\displaystyle{#1}\hfil$\crcr}}}\limits}
\def\underparenthesis#1{\mathop{\vtop{\ialign{##\crcr
$\hfil\displaystyle{#1}\hfil$\crcr\noalign{\kern3\p@\nointerlineskip}
\upparenthfill\crcr\noalign{\kern3\p@}}}}\limits}
\def\downparenthfill{$\m@th\braceld\leaders\vrule\hfill\bracerd$}
\def\upparenthfill{$\m@th\bracelu\leaders\vrule\hfill\braceru$}
\def\upbracketfill{$\m@th\makesm@sh{\llap{\vrule\@height3\p@\@width.7\p@}}%
\leaders\vrule\@height.7\p@\hfill
\makesm@sh{\rlap{\vrule\@height3\p@\@width.7\p@}}$}
\def\downbracketfill{$\m@th
\makesm@sh{\llap{\vrule\@height.7\p@\@depth2.3\p@\@width.7\p@}}%
\leaders\vrule\@height.7\p@\hfill
\makesm@sh{\rlap{\vrule\@height.7\p@\@depth2.3\p@\@width.7\p@}}$}
\makeatother
%%%%%%%%%%%%%%%%%%%%%%%%%%%%%%%%%%%%

% Markierung als Studenten rechnen lassen>
\newcommand{\markstudent}[3] %Argumente: 1: Abstand zwischen jetziger Zeile und Inhalt (Bei Equations: Baselineskip)
% 2.: L�nge des Inhalts /Markierung
%3. : Inhalt
{
\ifLeiter
\begin{minipage}[t]{0.05\textwidth}	
	\vspace{#1}
	\unitlength1cm
	{\color{orange}
	\makebox[0.025\textwidth]{\line(0,-1){#2}} 
	\makebox[0.025\textwidth]{\line(0,-1){#2}}}
\end{minipage}
\begin{minipage}[t]{0.95\textwidth}
#3
\end{minipage}
\else
#3
\fi
}

% Markierung f�r �bungsleiter>
\newcommand{\markteacher}[3] %Argumente: 1: Abstand zwischen jetziger Zeile und Inhalt (Bei Equations: Baselineskip)
% 2.: L�nge des Inhalts /Markierung
%3. : Inhalt
{
\ifLeiter
\begin{minipage}[t]{0.05\textwidth}	
	\vspace{#1}
	\unitlength1cm
	{\color{ForestGreen}
	\makebox[0.025\textwidth]{\begin{sideways}\uwave{\hspace{#2}}\end{sideways}}
	\makebox[0.025\textwidth]{\begin{sideways}\uwave{\hspace{#2}}\end{sideways}}  
	}
\end{minipage}
\begin{minipage}[t]{0.95\textwidth}
#3
\end{minipage}
\else
#3
\fi
}

% Markierung f�r �bungsleiter, einzeilig!
\newcommand{\markteachers}[1] 
{
\ifLeiter
\begin{minipage}[t]{0.05\textwidth}	
	\vspace{-0.75\baselineskip}
	\unitlength1cm
	{\color{OliveGreen}
	\makebox[0.025\textwidth]{\begin{sideways}\uwave{\hspace{0.85\baselineskip}}\end{sideways}}
	\makebox[0.025\textwidth]{\begin{sideways}\uwave{\hspace{0.85\baselineskip}}\end{sideways}}  
	}
\end{minipage}
\hspace{-0.5cm}
\begin{minipage}[t]{0.95\textwidth}
#1%
\end{minipage}
\else
#1
\fi
}

% Abs�tze machen:
\newcommand{\absatz}[2]
{
\ifdefined\wortlang 
\else
\newlength{\wortlang}
\fi
\settowidth{\wortlang}{#1}
\ifdim \wortlang<0.5\linewidth
\begin{tabular}{p{\wortlang}p{\linewidth-4\tabcolsep-\wortlang}}
\else
\begin{tabular}{p{0.5\linewidth-2\tabcolsep}p{0.5\linewidth-2\tabcolsep}}
\fi
#1 & #2
\end{tabular}
\let\wortlang\relax
}




\newcommand{\koords}{
\put(0,0){\makebox(0,0)[rb]{
\begin{picture}(0,0)
\put(4,0){\line(0,0){0,1}$4$}
\put(3,0){\line(0,0){0,1}$3$}
\put(2,0){\line(0,0){0,1}$2$}
\put(1,0){\line(0,0){0,1}$1$}
\put(0,0){\line(0,0){0,1}$0$}
\put(-1,0){\line(0,0){0,1}$1$}
\put(-2,0){\line(0,0){0,1}$2$}
\put(-3,0){\line(0,0){0,1}$3$}
\put(-4,0){\line(0,0){0,1}$4$}
\put(-5,0){\line(0,0){0,1}$5$}
\put(-6,0){\line(0,0){0,1}$6$}
\put(-7,0){\line(0,0){0,1}$7$}
\put(-8,0){\line(0,0){0,1}$8$}
\put(-9,0){\line(0,0){0,1}$9$}
\put(-10,0){\line(0,0){0,1}$10$}
\put(-11,0){\line(0,0){0,1}$11$}
\put(-12,0){\line(0,0){0,1}$12$}
\put(-13,0){\line(0,0){0,1}$13$}
\put(-14,0){\line(0,0){0,1}$14$}
\put(-15,0){\line(0,0){0,1}$15$}
\put(-16,0){\line(0,0){0,1}$16$}
\put(-17,0){\line(0,0){0,1}$17$}
\put(-18,0){\line(0,0){0,1}$18$}
\put(-19,0){\line(0,0){0,1}$19$}
\put(-20,0){\line(0,0){0,1}$20$}
\put(-21,0){\line(0,0){0,1}$21$}
\put(-22,0){\line(0,0){0,1}$22$}
\put(-23,0){\line(0,0){0,1}$23$}
\put(-24,0){\line(0,0){0,1}$24$}
\put(0,-5){\line(-1,0){0,1}$5$}
\put(0,-4){\line(-1,0){0,1}$4$}
\put(0,-3){\line(-1,0){0,1}$3$}
\put(0,-2){\line(-1,0){0,1}$2$}
\put(0,-1){\line(-1,0){0,1}$1$}
\put(0,0){\line(-1,0){0,1}}
\put(0,1){\line(-1,0){0,1}$1$}
\put(0,2){\line(-1,0){0,1}$2$}
\put(0,3){\line(-1,0){0,1}$3$}
\put(0,4){\line(-1,0){0,1}$4$}
\put(0,5){\line(-1,0){0,1}$5$}
\put(0,6){\line(-1,0){0,1}$6$}
\put(0,7){\line(-1,0){0,1}$7$}
\put(0,8){\line(-1,0){0,1}$8$}
\put(0,9){\line(-1,0){0,1}$9$}
\put(0,10){\line(-1,0){0,1}$10$}
\put(0,11){\line(-1,0){0,1}$11$}
\put(0,12){\line(-1,0){0,1}$12$}
\put(0,13){\line(-1,0){0,1}$13$}
\put(0,14){\line(-1,0){0,1}$14$}
\put(0,15){\line(-1,0){0,1}$15$}
\put(0,16){\line(-1,0){0,1}$16$}
\end{picture}
}
}
}

\renewcommand{\L}{\mathrsfs{L}}

%####################################################################################
%
%\pgfplotsset{
%tick  label  style={font=\scriptsize},
%label  style={font=\scriptsize},
%legend  style={font=\scriptsize}
%}

\begin{document}
%\tikzset{%
%  highlight/.style={rectangle,rounded corners,draw=orange,very thin,fill opacity=0.5,inner sep=-2pt}
%}



\newlength{\breite}
\newlength{\hoehe}
\setlength{\breite}{0.95\textwidth}%{0.7\textwidth}
\setlength{\hoehe}{0.85\textheight}%{0.6\textheight}
% siehe RTB Texen makros: tikzmark um optionales Argument des inner sep erweitert!
\newcommand{\vergleich}[1]{
\addplot [
#1
]
table[row sep=crcr]{
0 1\\
0.2 0.818730773333333\\
0.4 0.670320079202998\\
0.6 0.548811676826732\\
0.8 0.449329008582714\\
1 0.367879486678025\\
1.2 0.301194256621369\\
1.4 0.246597006647172\\
1.6 0.201896557953924\\
1.8 0.165298925026955\\
2 0.135335316718487\\
2.2 0.110803188516239\\
2.4 0.0907179802216991\\
2.6 0.0742736021021498\\
2.8 0.0608100836873454\\
3 0.049787086843805\\
3.2 0.0407622201136423\\
3.4 0.0333732839964259\\
3.6 0.0273237346150667\\
3.8 0.0223707823717484\\
4 0.0183156479512932\\
4.2 0.0149955846112634\\
4.4 0.0122773465853651\\
4.6 0.0100518414643173\\
4.8 0.0082297519355046\\
5 0.00673795116649717\\
5.2 0.00551656796922847\\
5.4 0.00451658395959232\\
5.6 0.00369786627806195\\
5.8 0.00302755691752091\\
6 0.00247875401639258\\
6.2 0.0020294321927442\\
6.4 0.00166155858859302\\
6.6 0.00136036914817741\\
6.8 0.0011137760847061\\
7 0.000911882755151595\\
7.2 0.000746586473314596\\
7.4 0.000611253320657065\\
7.6 0.000500451903924127\\
7.8 0.000409735374315939\\
8 0.000335462959875712\\
8.2 0.000274653848563731\\
8.4 0.000224867557833559\\
8.6 0.000184105989522648\\
8.8 0.000150733239177176\\
9 0.000123409941478568\\
9.2 0.000101039516823769\\
9.4 8.27241617463509e-005\\
9.6 6.77288169199417e-005\\
9.8 5.54516666538156e-005\\
10 4.53999859221005e-005\\
};
}
\begin{frame}{euler,ex}

\definecolor{mycolor1}{rgb}{0,1,1}%
\definecolor{mycolor2}{rgb}{1,0,1}%
%
\begin{tikzpicture}

\only<1>{\begin{axis}[%
width=\breite,
height=\hoehe,
%scale only axis, % Angegebene Groesse bezieht sich auf die Achsen!!
xmin=0,
xmax=5,
xtick={0},
xmajorgrids,
ymin=0,
ymax=1,
ytick={0,1},
yticklabels={0,1},
ymajorgrids,
%axis x line*=bottom,
%axis y line*=left,
xlabel=$t$,
%ylabel=$x(t)$,
axis background/.style={fill=white}
%legend style={draw=black,legend cell align=left,anchor=south east,at={(1,0)}},
]}
\only<2>{\begin{axis}[%
width=\breite,
height=\hoehe,
%scale only axis, % Angegebene Groesse bezieht sich auf die Achsen!!
xmin=0,
xmax=5,
xmajorgrids,
xtick={0,0.5,1,1.5,2,2.5,3,3.5,4,4.5},
ymin=0,
ymax=1,
ytick={0,1},
yticklabels={0,1},
ymajorgrids,
%axis x line*=bottom,
%axis y line*=left,
xlabel=$t$,
%ylabel=$x(t)$,
axis background/.style={fill=white}
%legend style={draw=black,legend cell align=left,anchor=south east,at={(1,0)}},
]}
\only<3>{\begin{axis}[%
width=\breite,
height=\hoehe,
%scale only axis, % Angegebene Groesse bezieht sich auf die Achsen!!
xmin=0,
xmax=5,
xmajorgrids,
xtick={0,0.5,1,1.5,2,2.5,3,3.5,4,4.5},
ymin=0,
ymax=1,
ytick={0,0.5,1},
yticklabels={0,0.5,1},
ymajorgrids,
%axis x line*=bottom,
%axis y line*=left,
xlabel=$t$,
%ylabel=$x(t)$,
axis background/.style={fill=white}
%legend style={draw=black,legend cell align=left,anchor=south east,at={(1,0)}},
]}
\only<4-6>{\begin{axis}[%
width=\breite,
height=\hoehe,
%scale only axis, % Angegebene Groesse bezieht sich auf die Achsen!!
xmin=0,
xmax=5,
xmajorgrids,
xtick={0,0.5,1,1.5,2,2.5,3,3.5,4,4.5},
ymin=0,
ymax=1,
ytick={0,0.25,0.5,1},
yticklabels={0,0.25,0.5,1},
ymajorgrids,
%axis x line*=bottom,
%axis y line*=left,
xlabel=$t$,
%ylabel=$x(t)$,
axis background/.style={fill=white}
%legend style={draw=black,legend cell align=left,anchor=south east,at={(1,0)}},
]}
\only<7-9>{\begin{axis}[%
width=\breite,
height=\hoehe,
%scale only axis, % Angegebene Groesse bezieht sich auf die Achsen!!
xmin=0,
xmax=7,
xmajorgrids,
xtick={0,0.5,1,1.5,2,2.5,3,3.5,4,4.5},
ymin=-1.5,
ymax=1.5,
ytick={-1,-0.5,0,0.5,1},
yticklabels={-1,-0.5,0,0.5,1},
ymajorgrids,
%axis x line*=bottom,
%axis y line*=left,
xlabel=$t$,
%ylabel=$x(t)$,
axis background/.style={fill=white}
%legend style={draw=black,legend cell align=left,anchor=south east,at={(1,0)}},
]}
\only<1-2>{
\vergleich{color=black,solid, thick,forget plot}}
\only<3->{
\vergleich{color=gray,dashed, thin,forget plot}}
%\addlegendentry{\scriptsize Vergleich};
%\node(vergleichtext) at ($(vergleich)+(10,100)$){\scriptsize{Vergleichsl�sung}};
%\draw[black](vergleich)--(vergleichtext);

\only<6>{
\addplot [
color=blue,very thick,
solid
]
table[row sep=crcr]{
0 1\\
0.1 0.9\\
0.2 0.81\\
0.3 0.729\\
0.4 0.6561\\
0.5 0.59049\\
0.6 0.531441\\
0.7 0.4782969\\
0.8 0.43046721\\
0.9 0.387420489\\
1 0.3486784401\\
1.1 0.31381059609\\
1.2 0.282429536481\\
1.3 0.2541865828329\\
1.4 0.22876792454961\\
1.5 0.205891132094649\\
1.6 0.185302018885184\\
1.7 0.166771816996666\\
1.8 0.150094635296999\\
1.9 0.135085171767299\\
2 0.121576654590569\\
2.1 0.109418989131512\\
2.2 0.0984770902183611\\
2.3 0.088629381196525\\
2.4 0.0797664430768725\\
2.5 0.0717897987691852\\
2.6 0.0646108188922667\\
2.7 0.05814973700304\\
2.8 0.052334763302736\\
2.9 0.0471012869724624\\
3 0.0423911582752162\\
3.1 0.0381520424476946\\
3.2 0.0343368382029251\\
3.3 0.0309031543826326\\
3.4 0.0278128389443693\\
3.5 0.0250315550499324\\
3.6 0.0225283995449392\\
3.7 0.0202755595904453\\
3.8 0.0182480036314007\\
3.9 0.0164232032682607\\
4 0.0147808829414346\\
4.1 0.0133027946472911\\
4.2 0.011972515182562\\
4.3 0.0107752636643058\\
4.4 0.00969773729787523\\
4.5 0.00872796356808771\\
4.6 0.00785516721127894\\
4.7 0.00706965049015105\\
4.8 0.00636268544113594\\
4.9 0.00572641689702235\\
5 0.00515377520732011\\
5.1 0.0046383976865881\\
5.2 0.00417455791792929\\
5.3 0.00375710212613636\\
5.4 0.00338139191352273\\
5.5 0.00304325272217045\\
5.6 0.00273892744995341\\
5.7 0.00246503470495807\\
5.8 0.00221853123446226\\
5.9 0.00199667811101603\\
6 0.00179701029991443\\
6.1 0.00161730926992299\\
6.2 0.00145557834293069\\
6.3 0.00131002050863762\\
6.4 0.00117901845777386\\
6.5 0.00106111661199647\\
6.6 0.000955004950796825\\
6.7 0.000859504455717143\\
6.8 0.000773554010145428\\
6.9 0.000696198609130885\\
7 0.000626578748217797\\
7.1 0.000563920873396017\\
7.2 0.000507528786056415\\
7.3 0.000456775907450774\\
7.4 0.000411098316705696\\
7.5 0.000369988485035127\\
7.6 0.000332989636531614\\
7.7 0.000299690672878453\\
7.8 0.000269721605590607\\
7.9 0.000242749445031547\\
8 0.000218474500528392\\
8.1 0.000196627050475553\\
8.2 0.000176964345427998\\
8.3 0.000159267910885198\\
8.4 0.000143341119796678\\
8.5 0.00012900700781701\\
8.6 0.000116106307035309\\
8.7 0.000104495676331778\\
8.8 9.40461086986004e-005\\
8.9 8.46414978287404e-005\\
9 7.61773480458663e-005\\
9.1 6.85596132412797e-005\\
9.2 6.17036519171517e-005\\
9.3 5.55332867254366e-005\\
9.4 4.99799580528929e-005\\
9.5 4.49819622476036e-005\\
9.6 4.04837660228433e-005\\
9.7 3.64353894205589e-005\\
9.8 3.2791850478503e-005\\
9.9 2.95126654306527e-005\\
10 2.65613988875875e-005\\
};}
%%\addlegendentry{\scriptsize $T=0.1/|\lambda|$};
%

\only<3>{\addplot [
color=green, very thick,
solid
]
table[row sep=crcr]{
0 1\\
0.5 0.5\\};}
\only<4>{\addplot [
color=green, very thick,
solid
]
table[row sep=crcr]{
0 1\\
0.5 0.5\\
1 0.25\\
};}
\only<5-6>{\addplot [
color=green, very thick,
solid
]
table[row sep=crcr]{
0 1\\
0.5 0.5\\
1 0.25\\
1.5 0.125\\
2 0.0625\\
2.5 0.03125\\
3 0.015625\\
3.5 0.0078125\\
4 0.00390625\\
4.5 0.001953125\\
5 0.0009765625\\
5.5 0.00048828125\\
6 0.000244140625\\
6.5 0.0001220703125\\
7 6.103515625e-005\\
7.5 3.0517578125e-005\\
8 1.52587890625e-005\\
8.5 7.62939453125e-006\\
9 3.814697265625e-006\\
9.5 1.9073486328125e-006\\
10 9.5367431640625e-007\\
};}
%%\addlegendentry{\scriptsize $T=0.5/|\lambda|$};
%
%\addplot [
%color=blue,
%solid, thick
%]
%table[row sep=crcr]{
%0 1\\
%1 0\\
%2 0\\
%3 0\\
%4 0\\
%5 0\\
%6 0\\
%7 0\\
%8 0\\
%9 0\\
%10 0\\
%};
%%\addlegendentry{\scriptsize $T=1/|\lambda|$};
%
%\addplot [
%color=red,
%solid
%]
%table[row sep=crcr]{
%0 1\\
%2 -1\\
%4 1\\
%6 -1\\
%8 1\\
%10 -1\\
%};
%%\addlegendentry{\scriptsize $T=2/|\lambda|$};
\only<8->{
\addplot [
color=mycolor2,
solid, very thick
]
table[row sep=crcr]{
0 1\\
2.1 -1.1\\
};}
\only<9>{
\addplot [
color=mycolor2,
solid, very thick
]
table[row sep=crcr]{
0 1\\
2.1 -1.1\\
4.2 1.21\\
6.3 -1.331\\
8.4 1.4641\\
};}
%%\addlegendentry{\scriptsize $T=2.1/|\lambda|$};

\end{axis}
\end{tikzpicture}%
\end{frame}
\begin{frame}{Beispiel}
\ctikzset{bipoles/thickness=1}
\ctikzset{bipoles/length=0.8cm}
\ctikzset{bipoles/diode/height=.375}
\ctikzset{bipoles/diode/width=.3}
\ctikzset{tripoles/thyristor/height=.8}
\ctikzset{tripoles/thyristor/width=1}
\ctikzset{bipoles/vsourceam/height/.initial=.7}
\ctikzset{bipoles/vsourceam/width/.initial=.7}
\tikzstyle{every node}=[font=\small]
\tikzstyle{every path}=[line width=0.8pt,line cap=round,line join=round]

\only<1>{
\begin{circuitikz}[scale=1.8]
\draw (0,0)
	to [C,l=$C$,v_>=$u_C(t)$] (2,0)	
	to node[currarrow]{}node[above]{$i(t)$} (3,0)
	to (3,-1.5)
	to [R,l^=$R$,v_=$u_R(t)$] node (R){} (0,-1.5)
	to (0,0);
\end{circuitikz}}
\only<2->{
\begin{minipage}{0.7\textwidth}
\begin{align*}
u_C(t)&=u_R(t):=u(t)\\
i(t) &= \frac{-u_R(t)}{R}  \\
u_C(t) &= u_0 + \frac{1}{C} \int \limits_0^t i(t) \mathrm{d}t \\
\dot{u}(t) &= \tikzmark{oben}{$\frac{-1}{RC}$} u(t), \quad u(0) = u_0\\
\dot{x} &= \tikzmark[-0pt]{lambda}{$\lambda$}x, \quad x(0)=x_0\\
&\text{hier: } \dot{x}=-x
\end{align*}
Echte L�sung:\\
$x(t) = x_0 e^{\lambda t}$
\end{minipage}

\tikz[overlay, remember picture]{
\node(obenneu)[highlight,fit=(oben.north west)(oben.south east)]{}; 
\draw[orange, very thin] (obenneu)->(lambda);
}}
%%\begin{circuitikz}
%    \draw
%    (0,0)
%        to[V, l=$V_s$] ++(0,2.5)
%        to[] ++(1,0) coordinate (A)
%        to[short] ++(0.5,0)
%        to[L, l^=$L_1$, v=$v_{L_1}$] ++(1.5,0)
%        to[short] ++(1,0) coordinate (B)
%        to[short] ++(1,0) node[above] (C) {1}
%        to[open, o-o] ++(0.65,0) coordinate (D)
%        to[short] ++(0.5,0)
%        to[L, l^=$L_2$, v=$v_{L_2}$] ++(1.5,0)
%        to[short] ++(0.5,0) coordinate (E)
%        to[short] ++(1.5,0)
%        to[generic, v^=$~~V_o$] ++(0,-2.5)
%        --(0,0)
%    (A)                                         % Left of L1, top of switch A
%        to[short] ++(0,-1.5) node[left] {2}
%        to[open, o-o] ++(0,-0.5) node[left] {1}
%        |- (0,0)
%    (B)                                         % C1 connection starting from top
%        to[C, l=$C_1$] ++(0,-1.75) coordinate (Aaux)
%        -- ($(A |- Aaux) + (0.5,0)$)
%        to[short, o-] ++(-0.5,-0.15)
%    ($(C)!0.5!(D)$)                             % Switch B low connector
%        ++(0,-0.5) node[left] {2}
%        to[short, o-] ++(0,-0.1)
%        |- (0,0)
%    (D)                                         % Switch B blade
%        to[short] ++(-0.65, -0.1)
%    (E)                                         % C2 connection
%        to[C, l=$C_2$] ++(0,-2.5)
%    (B)                                         % Vc1
%        to[open, v=$V_{C_1}~~$] (Aaux)
%    ;
%\end{circuitikz}
\end{frame}
\begin{frame}{euler,im}
% This file was created by matlab2tikz v0.4.3.
% Copyright (c) 2008--2013, Nico Schlömer <nico.schloemer@gmail.com>
% All rights reserved.
% 
% The latest updates can be retrieved from
%   http://www.mathworks.com/matlabcentral/fileexchange/22022-matlab2tikz
% where you can also make suggestions and rate matlab2tikz.
% 
%
% defining custom colors
\definecolor{mycolor1}{rgb}{0,1,1}%
\definecolor{mycolor2}{rgb}{1,0,1}%
%
\begin{tikzpicture}
%\only<2>{\begin{axis}[%
%width=0.8\textwidth,
%height=0.6\textheight,
%%scale only axis, % Angegebene Groesse bezieht sich auf die Achsen!!
%xmin=0,
%xmax=5,
%xmajorgrids,
%xtick={0,0.5,1,1.5,2,2.5,3,3.5,4,4.5},
%ymin=0,
%ymax=1,
%ytick={0,1},
%yticklabels={0,1},
%ymajorgrids,
%axis x line*=bottom,
%axis y line*=left,
%xlabel=$t$,
%ylabel=$x(t)$
%%legend style={draw=black,legend cell align=left,anchor=south east,at={(1,0)}},
%]}
\only<1>{\begin{axis}[%
width=\breite,
height=\hoehe,
%scale only axis,
xmin=0,
xmax=5,
xmajorgrids,
ytick={0,1},
xtick={0,0.5,1,1.5,2,2.5,3,3.5,4,4.5},
yticklabels={0,1},
ymin=-0.25,
ymax=1,
ymajorgrids,
xlabel=$t$,
%ylabel=$x(t)$,
axis background/.style={fill=white}
%legend style={draw=black,fill=white,legend cell align=left}
]}
\only<2-3>{\begin{axis}[%
width=\breite,
height=\hoehe,
%scale only axis,
xmin=0,
xmax=5,
xmajorgrids,
ytick={0,0.67,1},
yticklabels={0,0.67,1},
xtick={0,0.5,1,1.5,2,2.5,3,3.5,4,4.5},
ymin=-0.25,
ymax=1,
ymajorgrids,
xlabel=$t$,
%ylabel=$x(t)$,
axis background/.style={fill=white}
%legend style={draw=black,fill=white,legend cell align=left}
]}
\only<4-5>{\begin{axis}[%
width=\breite,
height=\hoehe,
%scale only axis,
xmin=0,
xmax=5,
xmajorgrids,
ytick={0,1},
yticklabels={0,1},
xtick={0,0.5,1,1.5,2,2.5,3,3.5,4,4.5},
ymin=-0.25,
ymax=1,
ymajorgrids,
xlabel=$t$,
%ylabel=$x(t)$,
axis background/.style={fill=white}
%legend style={draw=black,fill=white,legend cell align=left}
]}
\vergleich{color=gray,dashed, thin,forget plot}
\only<2>{
\addplot [
color=blue,
solid, very thick
]
table[row sep=crcr]{
0 1\\
0.5 0.666666666666667\\
};}

\only<3,4,5>{
\addplot [
color=blue,
solid, very thick
]
table[row sep=crcr]{
0 1\\
0.5 0.666666666666667\\
1 0.444444444444444\\
1.5 0.296296296296296\\
2 0.197530864197531\\
2.5 0.131687242798354\\
3 0.0877914951989026\\
3.5 0.0585276634659351\\
4 0.0390184423106234\\
4.5 0.0260122948737489\\
5 0.0173415299158326\\
5.5 0.0115610199438884\\
6 0.00770734662925894\\
6.5 0.00513823108617263\\
7 0.00342548739078175\\
7.5 0.00228365826052117\\
8 0.00152243884034745\\
8.5 0.0010149592268983\\
9 0.000676639484598865\\
9.5 0.000451092989732576\\
10 0.000300728659821718\\
10.5 0.000200485773214478\\
11 0.000133657182142986\\
11.5 8.91047880953237e-005\\
12 5.94031920635491e-005\\
};}
\only<4>{\addlegendentry{$T = 0.5$};}
\only<5>{\addlegendentry{impliziter Euler};}


\only<5>{\addplot [
color=green, very thick,
solid
]
table[row sep=crcr]{
0 1\\
0.5 0.5\\
1 0.25\\
1.5 0.125\\
2 0.0625\\
2.5 0.03125\\
3 0.015625\\
3.5 0.0078125\\
4 0.00390625\\
4.5 0.001953125\\
5 0.0009765625\\
5.5 0.00048828125\\
6 0.000244140625\\
6.5 0.0001220703125\\
7 6.103515625e-005\\
7.5 3.0517578125e-005\\
8 1.52587890625e-005\\
8.5 7.62939453125e-006\\
9 3.814697265625e-006\\
9.5 1.9073486328125e-006\\
10 9.5367431640625e-007\\
};}
\only<5>{\addlegendentry{expliziter Euler};}


\only<4>{\addplot [
color=mycolor2,
solid, very thick
]
table[row sep=crcr]{
0 1\\
2.1 0.32258064516129\\
4.2 0.104058272632674\\
6.3 0.0335671847202175\\
8.4 0.010828124103296\\
};}
\only<4>{\addlegendentry{$T=2.1$};}
%\addplot [
%color=green,
%solid
%]
%table[row sep=crcr]{
%0 1\\
%1 0.5\\
%2 0.25\\
%3 0.125\\
%4 0.0625\\
%5 0.03125\\
%6 0.015625\\
%7 0.0078125\\
%8 0.00390625\\
%9 0.001953125\\
%10 0.0009765625\\
%11 0.00048828125\\
%12 0.000244140625\\
%};
%%\addlegendentry{1};
%
%\addplot [
%color=blue,
%solid
%]
%table[row sep=crcr]{
%0 1\\
%2 0.333333333333333\\
%4 0.111111111111111\\
%6 0.037037037037037\\
%8 0.0123456790123457\\
%10 0.00411522633744856\\
%12 0.00137174211248285\\
%};
%%\addlegendentry{2};
%
%\addplot [
%color=mycolor1,
%solid
%]
%table[row sep=crcr]{
%0 1\\
%4 0.2\\
%8 0.04\\
%12 0.008\\
%};
%\addlegendentry{4};

\end{axis}
\end{tikzpicture}%
\end{frame}
\begin{frame}{Trapez}
% This file was created by matlab2tikz v0.4.3.
% Copyright (c) 2008--2013, Nico Schlömer <nico.schloemer@gmail.com>
% All rights reserved.
% 
% The latest updates can be retrieved from
%   http://www.mathworks.com/matlabcentral/fileexchange/22022-matlab2tikz
% where you can also make suggestions and rate matlab2tikz.
% 
%
% defining custom colors
\definecolor{mycolor1}{rgb}{0,1,1}%
%
\begin{tikzpicture}

\only<1->{
\begin{axis}[%
width=0.8\textwidth,
height=0.6\textheight,
%scale only axis,
xmin=0,
xmax=5,
xmajorgrids,
ytick={0,1},
xtick={0,0.5,1,1.5,2,2.5,3,3.5,4,4.5},
yticklabels={0,1},
ymin=-0.25,
ymax=1,
ymajorgrids,
axis x line*=bottom,
axis y line*=left,
xlabel=$t$,
ylabel=$x(t)$
%legend style={draw=black,fill=white,legend cell align=left}
]}

\vergleich{color=gray,dashed, thin,forget plot}
\addplot [
color=red,
solid
]
table[row sep=crcr]{
0 1\\
0.5 0.6\\
1 0.36\\
1.5 0.216\\
2 0.1296\\
2.5 0.07776\\
3 0.046656\\
3.5 0.0279936\\
4 0.01679616\\
4.5 0.010077696\\
5 0.0060466176\\
5.5 0.00362797056\\
6 0.002176782336\\
6.5 0.0013060694016\\
7 0.00078364164096\\
7.5 0.000470184984576\\
8 0.0002821109907456\\
8.5 0.00016926659444736\\
9 0.000101559956668416\\
9.5 6.09359740010496e-005\\
10 3.65615844006297e-005\\
10.5 2.19369506403778e-005\\
11 1.31621703842267e-005\\
11.5 7.89730223053603e-006\\
12 4.73838133832162e-006\\
};
\addlegendentry{0.5};

\only<2->{
\addplot [
color=green,
solid
]
table[row sep=crcr]{
0 1\\
1 0.333333333333333\\
2 0.111111111111111\\
3 0.037037037037037\\
4 0.0123456790123457\\
5 0.00411522633744856\\
6 0.00137174211248285\\
7 0.000457247370827618\\
8 0.000152415790275873\\
9 5.08052634252909e-005\\
10 1.69350878084303e-005\\
11 5.64502926947676e-006\\
12 1.88167642315892e-006\\
};
\addlegendentry{1};}
%

%\addplot [
%color=mycolor1,
%solid
%]
%table[row sep=crcr]{
%0 1\\
%4 -0.333333333333333\\
%8 0.111111111111111\\
%12 -0.037037037037037\\
%};
%\addlegendentry{4};
%
\only<3->{
\addplot [
color=blue,
solid
]
table[row sep=crcr]{
0 1\\
2.1 -0.0243902439024391\\
4.2 0.000594883997620465\\
6.3 -1.45093657956211e-005\\
8.4 3.53886970624906e-007\\
};
\addlegendentry{2.1};}

\end{axis}
\end{tikzpicture}%
\end{frame}
\begin{frame}{Heun}
% This file was created by matlab2tikz v0.4.3.
% Copyright (c) 2008--2013, Nico Schlömer <nico.schloemer@gmail.com>
% All rights reserved.
% 
% The latest updates can be retrieved from
%   http://www.mathworks.com/matlabcentral/fileexchange/22022-matlab2tikz
% where you can also make suggestions and rate matlab2tikz.
% 
%
% defining custom colors
\definecolor{mycolor1}{rgb}{0,1,1}%
\definecolor{mycolor2}{rgb}{1,0,1}%
%
\begin{tikzpicture}

\begin{axis}[%
width=\breite,
height=\hoehe,
%scale only axis,
xmin=0,
xmax=5,
xmajorgrids,
ytick={0,1},
xtick={0,0.5,1,1.5,2,2.5,3,3.5,4,4.5},
yticklabels={0,1},
ymin=-0.25,
ymax=1.25,
ymajorgrids,
xlabel=.,
%ylabel=$x(t)$,
axis background/.style={fill=white},
legend style={draw=black,fill=white,legend cell align=left, at={(0.99,0.8)}}
]
\vergleich{color=gray,dashed, thin,forget plot}

%\addplot [
%color=mycolor1,
%solid
%]
%table[row sep=crcr]{
%0 1\\
%0.1 0.905\\
%0.2 0.819025\\
%0.3 0.741217625\\
%0.4 0.670801950625\\
%0.5 0.607075765315625\\
%0.6 0.549403567610641\\
%0.7 0.49721022868763\\
%0.8 0.449975256962305\\
%0.9 0.407227607550886\\
%1 0.368540984833552\\
%1.1 0.333529591274364\\
%1.2 0.3018442801033\\
%1.3 0.273169073493486\\
%1.4 0.247218011511605\\
%1.5 0.223732300418003\\
%1.6 0.202477731878292\\
%1.7 0.183242347349855\\
%1.8 0.165834324351618\\
%1.9 0.150080063538215\\
%2 0.135822457502084\\
%2.1 0.122919324039386\\
%2.2 0.111241988255645\\
%2.3 0.100673999371358\\
%2.4 0.0911099694310793\\
%2.5 0.0824545223351268\\
%2.6 0.0746213427132897\\
%2.7 0.0675323151555272\\
%2.8 0.0611167452157521\\
%2.9 0.0553106544202557\\
%3 0.0500561422503314\\
%3.1 0.0453008087365499\\
%3.2 0.0409972319065776\\
%3.3 0.0371024948754528\\
%3.4 0.0335777578622848\\
%3.5 0.0303878708653677\\
%3.6 0.0275010231331578\\
%3.7 0.0248884259355078\\
%3.8 0.0225240254716345\\
%3.9 0.0203842430518293\\
%4 0.0184477399619055\\
%4.1 0.0166952046655245\\
%4.2 0.0151091602222996\\
%4.3 0.0136737900011812\\
%4.4 0.012374779951069\\
%4.5 0.0111991758557174\\
%4.6 0.0101352541494243\\
%4.7 0.00917240500522895\\
%4.8 0.0083010265297322\\
%4.9 0.00751242900940764\\
%5 0.00679874825351392\\
%5.1 0.00615286716943009\\
%5.2 0.00556834478833423\\
%5.3 0.00503935203344248\\
%5.4 0.00456061359026545\\
%5.5 0.00412735529919023\\
%5.6 0.00373525654576716\\
%5.7 0.00338040717391928\\
%5.8 0.00305926849239695\\
%5.9 0.00276863798561924\\
%6 0.00250561737698541\\
%6.1 0.00226758372617179\\
%6.2 0.00205216327218547\\
%6.3 0.00185720776132785\\
%6.4 0.00168077302400171\\
%6.5 0.00152109958672155\\
%6.6 0.001376595125983\\
%6.7 0.00124581858901461\\
%6.8 0.00112746582305823\\
%6.9 0.00102035656986769\\
%7 0.000923422695730263\\
%7.1 0.000835697539635888\\
%7.2 0.000756306273370479\\
%7.3 0.000684457177400283\\
%7.4 0.000619433745547256\\
%7.5 0.000560587539720267\\
%7.6 0.000507331723446842\\
%7.7 0.000459135209719392\\
%7.8 0.00041551736479605\\
%7.9 0.000376043215140425\\
%8 0.000340319109702085\\
%8.1 0.000307988794280387\\
%8.2 0.00027872985882375\\
%8.3 0.000252250522235494\\
%8.4 0.000228286722623122\\
%8.5 0.000206599483973925\\
%8.6 0.000186972532996402\\
%8.7 0.000169210142361744\\
%8.8 0.000153135178837378\\
%8.9 0.000138587336847827\\
%9 0.000125421539847284\\
%9.1 0.000113506493561792\\
%9.2 0.000102723376673422\\
%9.3 9.29646558894466e-005\\
%9.4 8.41330135799491e-005\\
%9.5 7.6140377289854e-005\\
%9.6 6.89070414473178e-005\\
%9.7 6.23608725098226e-005\\
%9.8 5.64365896213895e-005\\
%9.9 5.10751136073575e-005\\
%10 4.62229778146585e-005\\
%};
%\addlegendentry{\mathrm{T}=0.1/l};
\addplot [
color=red,
solid, very thick
]
table[row sep=crcr]{
0 1\\
0.5 0.6\\
1 0.36\\
1.5 0.216\\
2 0.1296\\
2.5 0.07776\\
3 0.046656\\
3.5 0.0279936\\
4 0.01679616\\
4.5 0.010077696\\
5 0.0060466176\\
5.5 0.00362797056\\
6 0.002176782336\\
6.5 0.0013060694016\\
7 0.00078364164096\\
7.5 0.000470184984576\\
8 0.0002821109907456\\
8.5 0.00016926659444736\\
9 0.000101559956668416\\
9.5 6.09359740010496e-005\\
10 3.65615844006297e-005\\
10.5 2.19369506403778e-005\\
11 1.31621703842267e-005\\
11.5 7.89730223053603e-006\\
12 4.73838133832162e-006\\
};
\addlegendentry{Trapez, $\mathrm{T}=0.5$};

\addplot [
color=green,
solid,very thick
]
table[row sep=crcr]{
0 1\\
0.5 0.625\\
1 0.390625\\
1.5 0.244140625\\
2 0.152587890625\\
2.5 0.095367431640625\\
3 0.0596046447753906\\
3.5 0.0372529029846191\\
4 0.023283064365387\\
4.5 0.0145519152283669\\
5 0.00909494701772928\\
5.5 0.0056843418860808\\
6 0.0035527136788005\\
6.5 0.00222044604925031\\
7 0.00138777878078145\\
7.5 0.000867361737988404\\
8 0.000542101086242752\\
8.5 0.00033881317890172\\
9 0.000211758236813575\\
9.5 0.000132348898008484\\
10 8.27180612553028e-005\\
};
\addlegendentry{Heun, $\mathrm{T}=0.5$};

%\addplot [
%color=blue,
%solid
%]
%table[row sep=crcr]{
%0 1\\
%1 0.5\\
%2 0.25\\
%3 0.125\\
%4 0.0625\\
%5 0.03125\\
%6 0.015625\\
%7 0.0078125\\
%8 0.00390625\\
%9 0.001953125\\
%10 0.0009765625\\
%};
%\addlegendentry{\mathrm{T}=1/l};
%
%\addplot [
%color=red,
%solid
%]
%table[row sep=crcr]{
%0 1\\
%2 1\\
%4 1\\
%6 1\\
%8 1\\
%10 1\\
%};
%\addlegendentry{\mathrm{T}=2/l};
%
\only<2->{\addplot [
color=mycolor2, very thick,
solid
]
table[row sep=crcr]{
0 1\\
2.1 1.105\\
4.2 1.221025\\
6.3 1.349232625\\
8.4 1.490902050625\\
};
\addlegendentry{Heun, $\mathrm{T}=2.1$};}

\end{axis}
\end{tikzpicture}%
\end{frame}
\end{document}