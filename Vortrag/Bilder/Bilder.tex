\documentclass[german,hyperref={pdfpagelabels=false,breaklinks=true}]{beamer}
\usepackage{etex}
\usepackage[ngerman]{babel}
\usepackage[ansinew]{inputenc}
\usepackage{amsmath,amssymb,amstext,amsbsy} % American-Maths-Standardized-Symbols
%\usepackage{trfsigns} % Laplace Zeichen
%\usepackage{exscale}  % Richte Skalierung von Integral- und Summenzeichen
\usepackage{graphicx} % Einbinden von Grafiken
\usepackage{multimedia}
\usepackage[arrow, matrix, curve]{xy}
\usepackage{pifont}
%\usepackage{arydshln}
\usepackage{tikz}
\usepackage{pgfplots}
\usepackage{calc}
\usepackage{xparse}
\usepackage{siunitx}
\usepackage[european,smartlabels]{circuitikz}
\usetikzlibrary{intersections,decorations.pathreplacing,fit,calc,positioning,plothandlers,plotmarks,patterns}
\newlength{\hatchspread}
\newlength{\hatchthickness}
\newlength{\hatchshift}
\newcommand{\hatchcolor}{}
% declaring the keys in tikz
\tikzset{hatchspread/.code={\setlength{\hatchspread}{#1}},
         hatchthickness/.code={\setlength{\hatchthickness}{#1}},
         hatchshift/.code={\setlength{\hatchshift}{#1}},% must be >= 0
         hatchcolor/.code={\renewcommand{\hatchcolor}{#1}}}
% setting the default values
\tikzset{hatchspread=3pt,
         hatchthickness=0.4pt,
         hatchshift=0pt,% must be >= 0
         hatchcolor=black}
% declaring the pattern
\pgfdeclarepatternformonly[\hatchspread,\hatchthickness,\hatchshift,\hatchcolor]% variables
   {custom north west lines}% name
   {\pgfqpoint{\dimexpr-2\hatchthickness}{\dimexpr-2\hatchthickness}}% lower left corner
   {\pgfqpoint{\dimexpr\hatchspread+2\hatchthickness}{\dimexpr\hatchspread+2\hatchthickness}}% upper right corner
   {\pgfqpoint{\dimexpr\hatchspread}{\dimexpr\hatchspread}}% tile size
   {% shape description
    \pgfsetlinewidth{\hatchthickness}
    \pgfpathmoveto{\pgfqpoint{0pt}{\dimexpr\hatchspread+\hatchshift}}
    \pgfpathlineto{\pgfqpoint{\dimexpr\hatchspread+0.15pt+\hatchshift}{-0.15pt}}
    \ifdim \hatchshift > 0pt
      \pgfpathmoveto{\pgfqpoint{0pt}{\hatchshift}}
      \pgfpathlineto{\pgfqpoint{\dimexpr0.15pt+\hatchshift}{-0.15pt}}
    \fi
    \pgfsetstrokecolor{\hatchcolor}
%    \pgfsetdash{{1pt}{1pt}}{0pt}% dashing cannot work correctly in all situation this way
    \pgfusepath{stroke}
   }


\usetheme{default} %damit keine Kopfzeile, die viel Platz wegnimmt.
\usecolortheme[RGB={20, 30,100}]{structure}  %leichtes Blau, �hnlich dem FAU Logo
\useinnertheme{circles}  % Damit auch Gliederung sch�n nummeriert wird

\setbeamercovered{transparent}
\beamertemplatenavigationsymbolsempty %damit keine Navigationssymbole
\setbeamerfont{frametitle}{size=\large}
\setbeamersize{text margin left=10mm}
\setbeamersize{text margin right=5mm}

%####################################################################################
%
\begin{document}
\tikzset{%
  highlight/.style={rectangle,rounded corners,draw=orange,very thin,fill opacity=0.5,inner sep=-2pt}
}
% siehe RTB Texen makros: tikzmark um optionales Argument des inner sep erweitert!
\newcommand{\vergleich}[1]{
\addplot [
#1
]
table[row sep=crcr]{
0 1\\
0.2 0.818730773333333\\
0.4 0.670320079202998\\
0.6 0.548811676826732\\
0.8 0.449329008582714\\
1 0.367879486678025\\
1.2 0.301194256621369\\
1.4 0.246597006647172\\
1.6 0.201896557953924\\
1.8 0.165298925026955\\
2 0.135335316718487\\
2.2 0.110803188516239\\
2.4 0.0907179802216991\\
2.6 0.0742736021021498\\
2.8 0.0608100836873454\\
3 0.049787086843805\\
3.2 0.0407622201136423\\
3.4 0.0333732839964259\\
3.6 0.0273237346150667\\
3.8 0.0223707823717484\\
4 0.0183156479512932\\
4.2 0.0149955846112634\\
4.4 0.0122773465853651\\
4.6 0.0100518414643173\\
4.8 0.0082297519355046\\
5 0.00673795116649717\\
5.2 0.00551656796922847\\
5.4 0.00451658395959232\\
5.6 0.00369786627806195\\
5.8 0.00302755691752091\\
6 0.00247875401639258\\
6.2 0.0020294321927442\\
6.4 0.00166155858859302\\
6.6 0.00136036914817741\\
6.8 0.0011137760847061\\
7 0.000911882755151595\\
7.2 0.000746586473314596\\
7.4 0.000611253320657065\\
7.6 0.000500451903924127\\
7.8 0.000409735374315939\\
8 0.000335462959875712\\
8.2 0.000274653848563731\\
8.4 0.000224867557833559\\
8.6 0.000184105989522648\\
8.8 0.000150733239177176\\
9 0.000123409941478568\\
9.2 0.000101039516823769\\
9.4 8.27241617463509e-005\\
9.6 6.77288169199417e-005\\
9.8 5.54516666538156e-005\\
10 4.53999859221005e-005\\
};
}
\begin{frame}

\definecolor{mycolor1}{rgb}{0,1,1}%
\definecolor{mycolor2}{rgb}{1,0,1}%
%
\begin{tikzpicture}

\only<1>{\begin{axis}[%
width=\breite,
height=\hoehe,
%scale only axis, % Angegebene Groesse bezieht sich auf die Achsen!!
xmin=0,
xmax=5,
xtick={0},
xmajorgrids,
ymin=0,
ymax=1,
ytick={0,1},
yticklabels={0,1},
ymajorgrids,
%axis x line*=bottom,
%axis y line*=left,
xlabel=$t$,
%ylabel=$x(t)$,
axis background/.style={fill=white}
%legend style={draw=black,legend cell align=left,anchor=south east,at={(1,0)}},
]}
\only<2>{\begin{axis}[%
width=\breite,
height=\hoehe,
%scale only axis, % Angegebene Groesse bezieht sich auf die Achsen!!
xmin=0,
xmax=5,
xmajorgrids,
xtick={0,0.5,1,1.5,2,2.5,3,3.5,4,4.5},
ymin=0,
ymax=1,
ytick={0,1},
yticklabels={0,1},
ymajorgrids,
%axis x line*=bottom,
%axis y line*=left,
xlabel=$t$,
%ylabel=$x(t)$,
axis background/.style={fill=white}
%legend style={draw=black,legend cell align=left,anchor=south east,at={(1,0)}},
]}
\only<3>{\begin{axis}[%
width=\breite,
height=\hoehe,
%scale only axis, % Angegebene Groesse bezieht sich auf die Achsen!!
xmin=0,
xmax=5,
xmajorgrids,
xtick={0,0.5,1,1.5,2,2.5,3,3.5,4,4.5},
ymin=0,
ymax=1,
ytick={0,0.5,1},
yticklabels={0,0.5,1},
ymajorgrids,
%axis x line*=bottom,
%axis y line*=left,
xlabel=$t$,
%ylabel=$x(t)$,
axis background/.style={fill=white}
%legend style={draw=black,legend cell align=left,anchor=south east,at={(1,0)}},
]}
\only<4-6>{\begin{axis}[%
width=\breite,
height=\hoehe,
%scale only axis, % Angegebene Groesse bezieht sich auf die Achsen!!
xmin=0,
xmax=5,
xmajorgrids,
xtick={0,0.5,1,1.5,2,2.5,3,3.5,4,4.5},
ymin=0,
ymax=1,
ytick={0,0.25,0.5,1},
yticklabels={0,0.25,0.5,1},
ymajorgrids,
%axis x line*=bottom,
%axis y line*=left,
xlabel=$t$,
%ylabel=$x(t)$,
axis background/.style={fill=white}
%legend style={draw=black,legend cell align=left,anchor=south east,at={(1,0)}},
]}
\only<7-9>{\begin{axis}[%
width=\breite,
height=\hoehe,
%scale only axis, % Angegebene Groesse bezieht sich auf die Achsen!!
xmin=0,
xmax=7,
xmajorgrids,
xtick={0,0.5,1,1.5,2,2.5,3,3.5,4,4.5},
ymin=-1.5,
ymax=1.5,
ytick={-1,-0.5,0,0.5,1},
yticklabels={-1,-0.5,0,0.5,1},
ymajorgrids,
%axis x line*=bottom,
%axis y line*=left,
xlabel=$t$,
%ylabel=$x(t)$,
axis background/.style={fill=white}
%legend style={draw=black,legend cell align=left,anchor=south east,at={(1,0)}},
]}
\only<1-2>{
\vergleich{color=black,solid, thick,forget plot}}
\only<3->{
\vergleich{color=gray,dashed, thin,forget plot}}
%\addlegendentry{\scriptsize Vergleich};
%\node(vergleichtext) at ($(vergleich)+(10,100)$){\scriptsize{Vergleichsl�sung}};
%\draw[black](vergleich)--(vergleichtext);

\only<6>{
\addplot [
color=blue,very thick,
solid
]
table[row sep=crcr]{
0 1\\
0.1 0.9\\
0.2 0.81\\
0.3 0.729\\
0.4 0.6561\\
0.5 0.59049\\
0.6 0.531441\\
0.7 0.4782969\\
0.8 0.43046721\\
0.9 0.387420489\\
1 0.3486784401\\
1.1 0.31381059609\\
1.2 0.282429536481\\
1.3 0.2541865828329\\
1.4 0.22876792454961\\
1.5 0.205891132094649\\
1.6 0.185302018885184\\
1.7 0.166771816996666\\
1.8 0.150094635296999\\
1.9 0.135085171767299\\
2 0.121576654590569\\
2.1 0.109418989131512\\
2.2 0.0984770902183611\\
2.3 0.088629381196525\\
2.4 0.0797664430768725\\
2.5 0.0717897987691852\\
2.6 0.0646108188922667\\
2.7 0.05814973700304\\
2.8 0.052334763302736\\
2.9 0.0471012869724624\\
3 0.0423911582752162\\
3.1 0.0381520424476946\\
3.2 0.0343368382029251\\
3.3 0.0309031543826326\\
3.4 0.0278128389443693\\
3.5 0.0250315550499324\\
3.6 0.0225283995449392\\
3.7 0.0202755595904453\\
3.8 0.0182480036314007\\
3.9 0.0164232032682607\\
4 0.0147808829414346\\
4.1 0.0133027946472911\\
4.2 0.011972515182562\\
4.3 0.0107752636643058\\
4.4 0.00969773729787523\\
4.5 0.00872796356808771\\
4.6 0.00785516721127894\\
4.7 0.00706965049015105\\
4.8 0.00636268544113594\\
4.9 0.00572641689702235\\
5 0.00515377520732011\\
5.1 0.0046383976865881\\
5.2 0.00417455791792929\\
5.3 0.00375710212613636\\
5.4 0.00338139191352273\\
5.5 0.00304325272217045\\
5.6 0.00273892744995341\\
5.7 0.00246503470495807\\
5.8 0.00221853123446226\\
5.9 0.00199667811101603\\
6 0.00179701029991443\\
6.1 0.00161730926992299\\
6.2 0.00145557834293069\\
6.3 0.00131002050863762\\
6.4 0.00117901845777386\\
6.5 0.00106111661199647\\
6.6 0.000955004950796825\\
6.7 0.000859504455717143\\
6.8 0.000773554010145428\\
6.9 0.000696198609130885\\
7 0.000626578748217797\\
7.1 0.000563920873396017\\
7.2 0.000507528786056415\\
7.3 0.000456775907450774\\
7.4 0.000411098316705696\\
7.5 0.000369988485035127\\
7.6 0.000332989636531614\\
7.7 0.000299690672878453\\
7.8 0.000269721605590607\\
7.9 0.000242749445031547\\
8 0.000218474500528392\\
8.1 0.000196627050475553\\
8.2 0.000176964345427998\\
8.3 0.000159267910885198\\
8.4 0.000143341119796678\\
8.5 0.00012900700781701\\
8.6 0.000116106307035309\\
8.7 0.000104495676331778\\
8.8 9.40461086986004e-005\\
8.9 8.46414978287404e-005\\
9 7.61773480458663e-005\\
9.1 6.85596132412797e-005\\
9.2 6.17036519171517e-005\\
9.3 5.55332867254366e-005\\
9.4 4.99799580528929e-005\\
9.5 4.49819622476036e-005\\
9.6 4.04837660228433e-005\\
9.7 3.64353894205589e-005\\
9.8 3.2791850478503e-005\\
9.9 2.95126654306527e-005\\
10 2.65613988875875e-005\\
};}
%%\addlegendentry{\scriptsize $T=0.1/|\lambda|$};
%

\only<3>{\addplot [
color=green, very thick,
solid
]
table[row sep=crcr]{
0 1\\
0.5 0.5\\};}
\only<4>{\addplot [
color=green, very thick,
solid
]
table[row sep=crcr]{
0 1\\
0.5 0.5\\
1 0.25\\
};}
\only<5-6>{\addplot [
color=green, very thick,
solid
]
table[row sep=crcr]{
0 1\\
0.5 0.5\\
1 0.25\\
1.5 0.125\\
2 0.0625\\
2.5 0.03125\\
3 0.015625\\
3.5 0.0078125\\
4 0.00390625\\
4.5 0.001953125\\
5 0.0009765625\\
5.5 0.00048828125\\
6 0.000244140625\\
6.5 0.0001220703125\\
7 6.103515625e-005\\
7.5 3.0517578125e-005\\
8 1.52587890625e-005\\
8.5 7.62939453125e-006\\
9 3.814697265625e-006\\
9.5 1.9073486328125e-006\\
10 9.5367431640625e-007\\
};}
%%\addlegendentry{\scriptsize $T=0.5/|\lambda|$};
%
%\addplot [
%color=blue,
%solid, thick
%]
%table[row sep=crcr]{
%0 1\\
%1 0\\
%2 0\\
%3 0\\
%4 0\\
%5 0\\
%6 0\\
%7 0\\
%8 0\\
%9 0\\
%10 0\\
%};
%%\addlegendentry{\scriptsize $T=1/|\lambda|$};
%
%\addplot [
%color=red,
%solid
%]
%table[row sep=crcr]{
%0 1\\
%2 -1\\
%4 1\\
%6 -1\\
%8 1\\
%10 -1\\
%};
%%\addlegendentry{\scriptsize $T=2/|\lambda|$};
\only<8->{
\addplot [
color=mycolor2,
solid, very thick
]
table[row sep=crcr]{
0 1\\
2.1 -1.1\\
};}
\only<9>{
\addplot [
color=mycolor2,
solid, very thick
]
table[row sep=crcr]{
0 1\\
2.1 -1.1\\
4.2 1.21\\
6.3 -1.331\\
8.4 1.4641\\
};}
%%\addlegendentry{\scriptsize $T=2.1/|\lambda|$};

\end{axis}
\end{tikzpicture}%
\end{frame}
\begin{frame}
\ctikzset{bipoles/thickness=1}
\ctikzset{bipoles/length=0.8cm}
\ctikzset{bipoles/diode/height=.375}
\ctikzset{bipoles/diode/width=.3}
\ctikzset{tripoles/thyristor/height=.8}
\ctikzset{tripoles/thyristor/width=1}
\ctikzset{bipoles/vsourceam/height/.initial=.7}
\ctikzset{bipoles/vsourceam/width/.initial=.7}
\tikzstyle{every node}=[font=\small]
\tikzstyle{every path}=[line width=0.8pt,line cap=round,line join=round]

\only<1>{
\begin{circuitikz}[scale=1.8]
\draw (0,0)
	to [C,l=$C$,v_>=$u_C(t)$] (2,0)	
	to node[currarrow]{}node[above]{$i(t)$} (3,0)
	to (3,-1.5)
	to [R,l^=$R$,v_=$u_R(t)$] node (R){} (0,-1.5)
	to (0,0);
\end{circuitikz}}
\only<2->{
\begin{minipage}{0.7\textwidth}
\begin{align*}
u_C(t)&=u_R(t):=u(t)\\
i(t) &= \frac{-u_R(t)}{R}  \\
u_C(t) &= u_0 + \frac{1}{C} \int \limits_0^t i(t) \mathrm{d}t \\
\dot{u}(t) &= \tikzmark{oben}{$\frac{-1}{RC}$} u(t), \quad u(0) = u_0\\
\dot{x} &= \tikzmark[-0pt]{lambda}{$\lambda$}x, \quad x(0)=x_0\\
&\text{hier: } \dot{x}=-x
\end{align*}
Echte L�sung:\\
$x(t) = x_0 e^{\lambda t}$
\end{minipage}

\tikz[overlay, remember picture]{
\node(obenneu)[highlight,fit=(oben.north west)(oben.south east)]{}; 
\draw[orange, very thin] (obenneu)->(lambda);
}}
%%\begin{circuitikz}
%    \draw
%    (0,0)
%        to[V, l=$V_s$] ++(0,2.5)
%        to[] ++(1,0) coordinate (A)
%        to[short] ++(0.5,0)
%        to[L, l^=$L_1$, v=$v_{L_1}$] ++(1.5,0)
%        to[short] ++(1,0) coordinate (B)
%        to[short] ++(1,0) node[above] (C) {1}
%        to[open, o-o] ++(0.65,0) coordinate (D)
%        to[short] ++(0.5,0)
%        to[L, l^=$L_2$, v=$v_{L_2}$] ++(1.5,0)
%        to[short] ++(0.5,0) coordinate (E)
%        to[short] ++(1.5,0)
%        to[generic, v^=$~~V_o$] ++(0,-2.5)
%        --(0,0)
%    (A)                                         % Left of L1, top of switch A
%        to[short] ++(0,-1.5) node[left] {2}
%        to[open, o-o] ++(0,-0.5) node[left] {1}
%        |- (0,0)
%    (B)                                         % C1 connection starting from top
%        to[C, l=$C_1$] ++(0,-1.75) coordinate (Aaux)
%        -- ($(A |- Aaux) + (0.5,0)$)
%        to[short, o-] ++(-0.5,-0.15)
%    ($(C)!0.5!(D)$)                             % Switch B low connector
%        ++(0,-0.5) node[left] {2}
%        to[short, o-] ++(0,-0.1)
%        |- (0,0)
%    (D)                                         % Switch B blade
%        to[short] ++(-0.65, -0.1)
%    (E)                                         % C2 connection
%        to[C, l=$C_2$] ++(0,-2.5)
%    (B)                                         % Vc1
%        to[open, v=$V_{C_1}~~$] (Aaux)
%    ;
%\end{circuitikz}
\end{frame}
\end{document}