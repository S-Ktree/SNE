\newcommand{\dd}{\cdot}
\renewcommand{\tt}{(t)}
\newcommand{\kk}{[k]}
\newcommand{\zz}{(z)}
\newcommand{\vs}{(s)}
\renewcommand{\u}[1]{u_#1(t)}
\newcommand{\y}[1]{y_#1(t)}
\newcommand{\x}[1]{x_#1(t)}
\newcommand{\itt}{\int_0^t}
\newcommand{\iti}{\int_0^\infty}
\newcommand{\etT}{\e^{-\frac{t}{T}}}
\newcommand{\ud}[1]{\underline{\dot{#1}}}
\newcommand{\ut}[1]{\underline{\tilde{#1}}}
\newcommand{\uh}[1]{\underline{\hat{#1}}}
\newcommand{\eps}{\ul{\varepsilon}}
\newcommand{\xk}{\underline{x}_k}
\newcommand{\ra}{\curvearrowright}
\newcommand{\eh}{\frac{1}{2}}
\newcommand{\ul}[1]{\underline{#1}}
\newcommand{\A}{\underline{A}}
\newcommand{\At}{\underline{\tilde{A}}}
\newcommand{\T}{\underline{T}}
\newcommand{\I}{\underline{I}}
\newcommand{\Pm}{\underline{P}}
\newcommand{\bv}{\underline{b}}
\newcommand{\bvt}{\underline{\tilde{b}}}
\newcommand{\xv}{\underline{x}}
\newcommand{\kt}{\ul{k}^T}
\newcommand{\ex}{\ul{e}_x}
\newcommand{\cv}{\underline{c}^T}
\newcommand{\cvt}{\underline{\tilde{c}}^T}
\newcommand{\ct}{\frac{c}{\Theta_M}}
\renewcommand{\d}{\mathrm{d}}
\newcommand{\jo}{\im\omega}
\newcommand{\ag}{\alpha\gamma}
\newcommand{\bg}{\beta\gamma}
\newcommand{\xvz}{\begin{bmatrix}
x_1\\x_2
\end{bmatrix}}
\newcommand{\xpz}{\begin{bmatrix}
\dot{x}_1\\ \dot{x}_2
\end{bmatrix}}
\newcommand{\PBZ}{\stackrel{\text{PBZ}}{=}}
\newcommand{\adj}{\mathrm{adj}}
%%%%%%%%%%%%%%%%%%%%%%%%%%
%TIKZ:
\tikzset{%
	>=latex,
  highlight/.style={rectangle,rounded corners,draw=orange,very thin,fill opacity=0.5,inner sep=-2pt},
  frame/.style={
	rectangle,minimum size=9mm,thick,draw=black
	},
 sum/.style={
	circle,minimum size =3mm,thick,draw=black, inner sep = 0pt
	},
 knot/.style={
	circle,minimum size =2mm,fill=black, inner sep=0pt
	},
 skip  loop/.style={to  path={--  ++(0,#1)  -|  (\tikztotarget)}},
 nonlin/.style = {frame, double,double distance=1pt},
 times/.style = {alias=sourcenode,frame, 
        append after command={
        ($(sourcenode.center)+(-0.2,0.2)$)--($(sourcenode.center)+(0.2,-0.2)$)
	  ($(sourcenode.center)+(-0.2,-0.2)$)--($(sourcenode.center)+(0.2,0.2)$)	
        }
	}
}



\newcommand{\tikzrightbrace}[4][0.5\textwidth]{%
%%% Ein rechter brace der die nodes 2 und 3 umschlie�t.
%%% rechts daneben wird der Text 4 mit der Breite 1 *optional angezeigt!
\tikz[overlay,remember picture]{
\node[fit=(#2.north east) (#3.south east),inner sep =0] (gesamt#2) {};
\draw[decorate, decoration=brace] (gesamt#2.north east)--(gesamt#2.south east);
\path (gesamt#2.east) node[right,xshift=2mm]{\parbox{#1}{#4}};
}
}


\newcommand{\tikzmark}[3][3pt]{\tikz[remember picture,baseline=(#2.base)] \node[inner sep=#1] (#2) {#3};}
%tikzmark: markiert inhalt als tikznode: 1(optional): inner sep der node
%									2: name der Node
%									3: Inhalt der Node
\newcommand{\tikzmarktwo}[3][2pt]{\tikz[remember picture,baseline=(#2.base)] 
\node[inner ysep=#1,inner xsep=0pt, outer xsep=0pt] (#2) {#3};}
%%tikzmark: markiert inhalt als tikznode: 1(optional): inner ysep der node
%%									2(optional): inner xsep der node
%%									3: name der Node
%%									4: Inhalt der Node

%% Befehl, um einfach Beschriftungen mit Pfeil hinzuzuf�gen: (h�lt Abstand nach unten/oben automatisch ein! (standardm��ig in footnotesize aber durch voranstellen einer gr��e �berschreibbar!)
% #1 (optional, def 0.5cm): Abstand der Beschriftung vom zu Beschriftenden
% #2 (optional, def 270): Winkel, in dem die Beschriftung angebracht wird.
% #3 zu beschriftender Text
% #4 Beschriftung 
\DeclareDocumentCommand \pin {O{0.5cm} O{270} m m}{
\tikz[baseline=(mynode.base), pin distance=#1,every pin edge/.style={<-}, inner sep=1pt]{
\node(mynode)[pin=#2:{\footnotesize  #4}]{#3};
\coordinate (pin) at (current bounding box.south);
\path let \p1=(pin), \p2=(mynode.center) in %
	node[inner sep=0pt](da) at (\x2,\y1){} node(gesamt)[fit=(mynode)(da),inner sep =0pt]{};
\pgfresetboundingbox
\useasboundingbox (gesamt.north west)rectangle(gesamt.south east);
%\draw[fill=red] (current bounding box.south west) circle(0.02cm);
%\draw[fill=red] (current bounding box.south east) circle(0.02cm);
}
}

\DeclareDocumentCommand \pintest{O{0.5cm} O{270} m m}{#1 #2 #3 #4}

\newcommand{\Highlight}[1][submatrix]{%
    \tikz[overlay,remember picture]{
    \node[highlight,fit=(left.north west) (right.south east)] (#1) {};}
}
\newcommand{\highlight}[2]{%
	\tikz[overlay,remember picture]{
	\node[highlight,fit=(#1.north west)(#1.south east)] (#2){};
	}
}
\newcommand{\underbrac}[2]{%
	\tikz[overlay,remember picture, very thin, orange]{
	\node[inner sep=-2pt,fit=(#1.north west)(#1.south east)] (#2){};
	\path (#1.south west)++(0.065,0.05) edge($(#1.south west)+(0.065,0)$)
		  ($(#1.south west)+(0.065,0)$) edge ($(#1.south east)-(0.065,0)$)
  	 	 ($(#1.south east)-(0.065,0)$)++(0,0.05) edge  ($(#1.south east)-(0.065,0)$);
	}
}
%%%%%%%%%%%%%%%%%%%%%%%%%%%%%%%%%%%%%%%
\newcommand{\tikzdoubleul}[1]{ % doppelte Unterstreichungen auch fuer groessere Objekte
\tikzmarktwo{tikzdoubleul}{#1}
\tikz[overlay, remember picture]{
\path (tikzdoubleul.south west) edge[double, thick, double distance = 1.5pt] (tikzdoubleul.south east);
}
}

\newcommand{\uldash}[1]{ % gestrichelte Unterstreichung
\tikzmarktwo[2pt]{uldash}{#1}
\tikz[overlay, remember picture]{
\path (uldash.south west) edge[draw, thick, dashed] (uldash.south east);
}
}

\newcommand{\tikzcancel}[1]{ % cancel gestrichelt
\tikzmarktwo{tikzcancel}{#1}
\tikz[overlay, remember picture]{
\path ($(tikzcancel.south west)-(1pt,2pt)$) edge[draw,dotted,thick,color =gray] ($(tikzcancel.north east)+(1pt,2pt)$);
}
}


%%%%%%%%%%%%%%%%%%%%%%%%%%%%%%
% fuer bessere Underbraces!!
% normal den underbrace Befehl verwenden aber dann den Text mathclapen! Bsp: \underbrace{a}{\mathclap{text}}
 \def\mathllap{\mathpalette\mathllapinternal}
 \def\mathllapinternal#1#2{%
 \llap{$\mathsurround=0pt#1{#2}$}%  $
 }
 \def\clap#1{\hbox  to  0pt{\hss#1\hss}}
 \def\mathclap{\mathpalette\mathclapinternal}
 \def\mathclapinternal#1#2{%
 \clap{$\mathsurround=0pt#1{#2}$}%
}
 \def\mathrlap{\mathpalette\mathrlapinternal}
 \def\mathrlapinternal#1#2{%
 \rlap{$\mathsurround=0pt#1{#2}$}%  $
 }
%%%%%%%%%%%%%%%%%%%%%%%%%%%%%%
%%%%%%%%%%%%%%%%%%%%%%%%%%%%
% underbrackets!
\makeatletter
\def\overbracket#1{\mathop{\vbox{\ialign{##\crcr\noalign{\kern3\p@}
\downbracketfill\crcr\noalign{\kern3\p@\nointerlineskip}
$\hfil\displaystyle{#1}\hfil$\crcr}}}\limits}
\def\underbracket#1{\mathop{\vtop{\ialign{##\crcr
$\hfil\displaystyle{#1}\hfil$\crcr\noalign{\kern3\p@\nointerlineskip}
\upbracketfill\crcr\noalign{\kern3\p@}}}}\limits}
\def\overparenthesis#1{\mathop{\vbox{\ialign{##\crcr\noalign{\kern3\p@}
\downparenthfill\crcr\noalign{\kern3\p@\nointerlineskip}
$\hfil\displaystyle{#1}\hfil$\crcr}}}\limits}
\def\underparenthesis#1{\mathop{\vtop{\ialign{##\crcr
$\hfil\displaystyle{#1}\hfil$\crcr\noalign{\kern3\p@\nointerlineskip}
\upparenthfill\crcr\noalign{\kern3\p@}}}}\limits}
\def\downparenthfill{$\m@th\braceld\leaders\vrule\hfill\bracerd$}
\def\upparenthfill{$\m@th\bracelu\leaders\vrule\hfill\braceru$}
\def\upbracketfill{$\m@th\makesm@sh{\llap{\vrule\@height3\p@\@width.7\p@}}%
\leaders\vrule\@height.7\p@\hfill
\makesm@sh{\rlap{\vrule\@height3\p@\@width.7\p@}}$}
\def\downbracketfill{$\m@th
\makesm@sh{\llap{\vrule\@height.7\p@\@depth2.3\p@\@width.7\p@}}%
\leaders\vrule\@height.7\p@\hfill
\makesm@sh{\rlap{\vrule\@height.7\p@\@depth2.3\p@\@width.7\p@}}$}
\makeatother
%%%%%%%%%%%%%%%%%%%%%%%%%%%%%%%%%%%%

% Markierung als Studenten rechnen lassen>
\newcommand{\markstudent}[3] %Argumente: 1: Abstand zwischen jetziger Zeile und Inhalt (Bei Equations: Baselineskip)
% 2.: L�nge des Inhalts /Markierung
%3. : Inhalt
{
\ifLeiter
\begin{minipage}[t]{0.05\textwidth}	
	\vspace{#1}
	\unitlength1cm
	{\color{orange}
	\makebox[0.025\textwidth]{\line(0,-1){#2}} 
	\makebox[0.025\textwidth]{\line(0,-1){#2}}}
\end{minipage}
\begin{minipage}[t]{0.95\textwidth}
#3
\end{minipage}
\else
#3
\fi
}

% Markierung f�r �bungsleiter>
\newcommand{\markteacher}[3] %Argumente: 1: Abstand zwischen jetziger Zeile und Inhalt (Bei Equations: Baselineskip)
% 2.: L�nge des Inhalts /Markierung
%3. : Inhalt
{
\ifLeiter
\begin{minipage}[t]{0.05\textwidth}	
	\vspace{#1}
	\unitlength1cm
	{\color{ForestGreen}
	\makebox[0.025\textwidth]{\begin{sideways}\uwave{\hspace{#2}}\end{sideways}}
	\makebox[0.025\textwidth]{\begin{sideways}\uwave{\hspace{#2}}\end{sideways}}  
	}
\end{minipage}
\begin{minipage}[t]{0.95\textwidth}
#3
\end{minipage}
\else
#3
\fi
}

% Markierung f�r �bungsleiter, einzeilig!
\newcommand{\markteachers}[1] 
{
\ifLeiter
\begin{minipage}[t]{0.05\textwidth}	
	\vspace{-0.75\baselineskip}
	\unitlength1cm
	{\color{OliveGreen}
	\makebox[0.025\textwidth]{\begin{sideways}\uwave{\hspace{0.85\baselineskip}}\end{sideways}}
	\makebox[0.025\textwidth]{\begin{sideways}\uwave{\hspace{0.85\baselineskip}}\end{sideways}}  
	}
\end{minipage}
\hspace{-0.5cm}
\begin{minipage}[t]{0.95\textwidth}
#1%
\end{minipage}
\else
#1
\fi
}

% Abs�tze machen:
\newcommand{\absatz}[2]
{
\ifdefined\wortlang 
\else
\newlength{\wortlang}
\fi
\settowidth{\wortlang}{#1}
\ifdim \wortlang<0.5\linewidth
\begin{tabular}{p{\wortlang}p{\linewidth-4\tabcolsep-\wortlang}}
\else
\begin{tabular}{p{0.5\linewidth-2\tabcolsep}p{0.5\linewidth-2\tabcolsep}}
\fi
#1 & #2
\end{tabular}
\let\wortlang\relax
}




\newcommand{\koords}{
\put(0,0){\makebox(0,0)[rb]{
\begin{picture}(0,0)
\put(4,0){\line(0,0){0,1}$4$}
\put(3,0){\line(0,0){0,1}$3$}
\put(2,0){\line(0,0){0,1}$2$}
\put(1,0){\line(0,0){0,1}$1$}
\put(0,0){\line(0,0){0,1}$0$}
\put(-1,0){\line(0,0){0,1}$1$}
\put(-2,0){\line(0,0){0,1}$2$}
\put(-3,0){\line(0,0){0,1}$3$}
\put(-4,0){\line(0,0){0,1}$4$}
\put(-5,0){\line(0,0){0,1}$5$}
\put(-6,0){\line(0,0){0,1}$6$}
\put(-7,0){\line(0,0){0,1}$7$}
\put(-8,0){\line(0,0){0,1}$8$}
\put(-9,0){\line(0,0){0,1}$9$}
\put(-10,0){\line(0,0){0,1}$10$}
\put(-11,0){\line(0,0){0,1}$11$}
\put(-12,0){\line(0,0){0,1}$12$}
\put(-13,0){\line(0,0){0,1}$13$}
\put(-14,0){\line(0,0){0,1}$14$}
\put(-15,0){\line(0,0){0,1}$15$}
\put(-16,0){\line(0,0){0,1}$16$}
\put(-17,0){\line(0,0){0,1}$17$}
\put(-18,0){\line(0,0){0,1}$18$}
\put(-19,0){\line(0,0){0,1}$19$}
\put(-20,0){\line(0,0){0,1}$20$}
\put(-21,0){\line(0,0){0,1}$21$}
\put(-22,0){\line(0,0){0,1}$22$}
\put(-23,0){\line(0,0){0,1}$23$}
\put(-24,0){\line(0,0){0,1}$24$}
\put(0,-5){\line(-1,0){0,1}$5$}
\put(0,-4){\line(-1,0){0,1}$4$}
\put(0,-3){\line(-1,0){0,1}$3$}
\put(0,-2){\line(-1,0){0,1}$2$}
\put(0,-1){\line(-1,0){0,1}$1$}
\put(0,0){\line(-1,0){0,1}}
\put(0,1){\line(-1,0){0,1}$1$}
\put(0,2){\line(-1,0){0,1}$2$}
\put(0,3){\line(-1,0){0,1}$3$}
\put(0,4){\line(-1,0){0,1}$4$}
\put(0,5){\line(-1,0){0,1}$5$}
\put(0,6){\line(-1,0){0,1}$6$}
\put(0,7){\line(-1,0){0,1}$7$}
\put(0,8){\line(-1,0){0,1}$8$}
\put(0,9){\line(-1,0){0,1}$9$}
\put(0,10){\line(-1,0){0,1}$10$}
\put(0,11){\line(-1,0){0,1}$11$}
\put(0,12){\line(-1,0){0,1}$12$}
\put(0,13){\line(-1,0){0,1}$13$}
\put(0,14){\line(-1,0){0,1}$14$}
\put(0,15){\line(-1,0){0,1}$15$}
\put(0,16){\line(-1,0){0,1}$16$}
\end{picture}
}
}
}

\renewcommand{\L}{\mathrsfs{L}}