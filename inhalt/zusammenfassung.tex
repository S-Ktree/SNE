\chapter{Zusammenfassung und Ausblick}
\label{kap:zusammenfassung}
In dieser Arbeit wurden die wichtigsten numerischen Integrationsverfahren f�r Anfangswertprobleme kurz vorgestellt. Dabei wurde sowohl auf explizite als auch implizite Verfahren eingegangen und zwischen Einschritt- und Mehrschrittverfahren unterschieden.
Die Vorgehensweise der Schrittweitensteuerung wurde dargelegt und deren Anwendung durch eingebettete Runge-Kutta-Verfahren beschrieben.\\
Dar�ber hinaus konnte eine allgemeine Vorgehensweise zur Betrachtung des Stabilit�tsverhaltens von Einschrittverfahren aufgezeigt werden und an einigen einfachen Verfahren verdeutlicht werden. Dabei wurden die Vor- und Nachteile der impliziten Verfahren gegen�ber den expliziten Verfahren deutlich.
F�r Mehrschrittverfahren konnte eine Methode zur �berpr�fung der Stabilit�t in konkreten Anwendungsf�llen vorgeschlagen werden.\\
Abschlie�end wurden ausgew�hlte, in der Energietechnik zum Einsatz kommende Computerprogramme knapp auf deren verwendete Integrationsverfahren untersucht.

Auf die Konstruktion der verschiedenen Verfahren wurde -- abgesehen von den einfachsten Verfahren -- im Rahmen dieser Arbeit bewusst verzichtet. Diese k�nnen in \cite{gruene} oder entsprechender Literatur nachgelesen werden. \\
Eine weitere Klasse von Verfahren, die sogenannten \textit{Extrapolationsverfahren}, stellen in ihrer Herleitung und Umsetzung eine besondere Form der allgemeinen Runge-Kutta-Verfahren dar und wurde im Zuge dieser Arbeit nicht n�her betrachtet.

 

