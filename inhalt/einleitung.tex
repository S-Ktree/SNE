\chapter{Einleitung}
\label{kap:einleitung}
Die Analyse von komplexen Energieversorgungsnetzen ist ohne das Hilfsmittel Computer kaum denkbar. Die Gr��e der Netze, sowie die Anzahl deren Komponenten und Speicherelemente macht eine rein analytische Berechnung sehr schwierig. Der Einsatz des Computers hat es dem Ingenieur nicht nur in der Energietechnik erm�glicht, verschiedenste dynamische Systeme zu simulieren, um dadurch Systemeigenschaften zu beobachten oder mit unerprobten Konzepten zu experimentieren. 
Dynamische Systeme sind dabei im allgemeinen dadurch gekennzeichnet, dass sie Energiespeicher enthalten, die zu zeitver�nderlichen Vorg�ngen f�hren. Aus physikalischer Sicht sind diese Vorg�nge durch Differentialgleichungen charakterisiert, deren L�sungen sich nicht immer in analytischer Form, oder zumindest einfach angeben lassen.
Der Computer liefert die M�glichkeit, mit numerischen Methoden N�herungen dieser L�sungen zu berechnen.
Hierf�r gibt es eine Vielzahl verschiedener Methoden, welche unter gewissen Voraussetzungen zu unterschiedlichen Ergebnissen oder zu unterschiedlicher Genauigkeit f�hren k�nnen. 
Um Simulationssoftware anwenden zu k�nnen ist es deshalb notwendig, dass der Benutzer die Grundlagen der hinter der Software steckenden Numerik versteht.

Aus diesem Grund werden in dieser Arbeit die wichtigsten Verfahren zu numerischen Simulation von dynamischen Systemen vorgestellt. Dabei wird auf Einschrittverfahren, sowie auf Mehrschrittverfahren eingegangen. Dar�ber hinaus wird die Vorgehensweise der Schrittweitensteuerung erl�utert.
Im \ref{sec:stabi}. Kapitel werden die verschiedenen Verfahren auf Stabilit�t untersucht, welche eine essenzielle Eigenschaft der numerischen Verfahren darstellt.
Um die Bedeutung der Verfahren zu verdeutlichen, werden in Kapitel \ref{sec:anwend} einige Softwarepakete der zur Netzsimulation auf ihr verwendetes Verfahren untersucht.
