\chapter{�berblick einiger Verfahren}
\section{Einschrittverfahren}
\begin{sideways}\centering
\begin{tabular}{|lcccc|}\hline
\textbf{Name}&\textbf{Ordnung}&\textbf{Butcher-Schema}&\textbf{Art}&\textbf{Stabilit�tsfunktion}\\ \hline
Euler &1 & $\begin{array}[c]{c|c}
 0 &  \\ \hline
 & 1
\end{array}$ & explizit & ...\\ \hdashline[1pt/2pt]
%%
Heun & 2 & $\begin{array}[c]{c|cc}
 0   \\
1 & 1 \\ \hline
 & \eh & \eh \end{array}$ & explizit & ... \\\hdashline[1pt/2pt]
%%%
klassisches Runge-Kutta & 4 & $\begin{array}[c]{c|cccc}
		0 &\\ 
		\eh&\eh \\ 
		\eh & 0 & \eh \\
		1&0&0&1 \\ \hline
		& \frac{1}{6}&\frac{2}{6}&\frac{2}{6}&\frac{1}{6}
	\end{array}$ & explizit & ...\\\hdashline[1pt/2pt]
%%%%%%%%%%%
Euler & 1 & $\begin{array}[c]{c|c}
		0 &\\ \hline
		&1
	\end{array}$ & implizit & ...\\\hdashline[1pt/2pt]
Mittelpunktsregel & 2 & $\begin{array}[c]{c|cc}
		0 &\\ 
		1&1 \\ \hline
		& \eh & \eh
	\end{array}$ & implizit & ... \\\hdashline[1pt/2pt]
Trapezregel & 2 &$\begin{array}[c]{c|cccc}
		0 &\\ 
		\eh&\eh \\ 
		\eh & 0 & \eh \\
		1&0&0&1 \\ \hline
		& \frac{1}{6}&\frac{2}{6}&\frac{2}{6}&\frac{1}{6}
	\end{array}$ & implizit & ... \\ \lasthline
\end{tabular}
\end{sideways}
%%%%%%%%%%%%%%%%%%%%%%%%%%%%%%%%%%%%%%%%%%%%%%
\newpage
\section{Eingebettete Runge-Kutta-Verfahren (Schrittweitensteuerung)}
\label{sec:anheingeb}
\begin{longtable}{|lcc|}\hline
\tikzmark[2pt]{1}{\parbox{2cm}{\textbf{Name/\\Ordnung}}}&\textbf{Stufen}&\textbf{Butcher-Schema}\\ \hline
\parbox{2cm}{Fehlberg-RK4(3)} & 5 & $
\begin{array}[c]{c|ccccc}
		0\\
		\eh & \eh \\
		\eh & 0 & \eh \\
		1 & 0 & 0 & 1 \\%
		1 & \frac{1}{6} & \frac{2}{6} & \frac{2}{6} & \frac{1}{6}\\\hline
		& \frac{1}{6} & \frac{2}{6} & \frac{2}{6} & \frac{1}{6} & 0 \\ \hline
		& \frac{1}{6} & \frac{2}{6} & \frac{2}{6} & 0 & \frac{1}{6} 
\end{array}
$ \\ \hdashline[1pt/2pt]
\parbox{2cm}{Dormand-Prince-RK5(4)} & 6 & $
\begin{array}[c]{c|ccccccc}
		0\\
		\frac{1}{5} & \frac{1}{5} \\
		\frac{3}{10} & \frac{3}{40} & \frac{9}{40} \\
		\frac{4}{5} & \frac{44}{45} & -\frac{56}{15} & \frac{32}{9} \\%
		\frac{8}{9} & \frac{19372}{6561} & -\frac{25360}{2187} & \frac{64448}{6561} & -\frac{5103}{18656}\\
		1 & \frac{9017}{3168} & -\frac{355}{33} & \frac{46732}{5247} & \frac{49}{176} & -\frac{5103}{18656}\\
		1 & \frac{35}{384} & 0 & \frac{500}{1113} & \frac{125}{92} & -\frac{2187}{6487} & \frac{11}{84}\\\hline
		& \frac{35}{384} & 0 & \frac{500}{1113} & \frac{125}{92} & -\frac{2187}{6487} & \frac{11}{84}&0 \\ \hline
		& \frac{5179}{57600} & 0 & \frac{7571}{16695} & \frac{393}{640} & -\frac{92097}{339200} & \frac{187}{2100} & \frac{1}{40} 
\end{array}
$ \\ \lasthline
\end{longtable}
%%%%%%%%%%%%%%%%%%%%%%%%%%%%%%%%%%%%%%%%%
\begin{landscape}
\section{Mehrschrittverfahren}
\label{sec:anhmehr}
\small
\arraycolsep0pt
\begin{longtable}{|llcccc|}\hline
$k$ &\textbf{Name}&\textbf{$P_a(z)$}&\textbf{$P_b(z)$}&\textbf{Art}&\textbf{Stabilit�t}\\ \hline
$\begin{array}{c}2\end{array}$ & \parbox{1.4cm}{Milne-Simpson} &  $z^2-1$ & $\frac{1}{3}z^2 + \frac{4}{3}z + \frac{1}{3}$ & implizit & ...? (Sinn?) \\[4pt] \hdashline[1pt/2pt]
$\begin{array}[c]{l}
1\\2\\3\\4
\end{array}$&
\parbox{1.4cm}{Adams-Bashforth} & $z^{k}-z^{k-1}$ &$
\begin{array}[c]{l}
	1 \ \text{\footnotesize($\hat{=}$ Euler, explizit)}\\
	(3z-1)/2 \\
	(23z^2-16z+5)/12\\
	(55z^3-59z^2+37z-9)/24
\end{array}$ & explizit & ... \\\hdashline[1pt/2pt]
$\begin{array}[c]{l}
2\\3\\4
\end{array}$&
\parbox{1.4cm}{Adams-Moulton} & $z^{k}-z^{k-1}$ &$
\begin{array}[c]{l}
	(5z^2+8z-1)/12\\
	(9z^3+19z^2-5z+1)/24\\
	(251z^4+646z^3+-246z^2+106z-19)
\end{array}$ & implizit & ... \\[4pt]\hdashline[1pt/2pt]
$\begin{array}[c]{l}
1\\2\\3\\4
\end{array}$&
BDF &%\makebox[2.5cm][l]{$\begin{array}[c]{l}
	$\begin{array}[c]{l}
	z-1 \ \text{\footnotesize($\hat{=}$ Euler, implizit)}\\
	\frac{3}{2}z^2-2z+\eh\\
	\frac{11}{6}z^3-3z^2+\frac{3}{2}z-\frac{1}{3}\\
	\frac{25}{12}z^4-4z^3+3z^2-\frac{4}{3}z+\frac{1}{4}\\
\end{array}$ & $z^k$ & implizit & \parbox{4cm}{unendlich gro�es Stabilit�tsgebiet $\Rightarrow$ gut f�r steife DGL} \\\lasthline
\end{longtable}
\end{landscape}

























