\chapter{Hauptteil}
\label{kap:hauptteil}
\section{Beschreibung der Integrationsverfahren}
Ausgangslage: Numerische L�sung von Anfangswertproblemen. Systeme beschrieben durch allgemeine Zustandsdarstellung:
(Zur Einfachheit Eingr��ensysteme.)
\begin{align}
\dot{\vec{x}}\tt&= \vec{A}\, \vec{x}\tt + \vec{b} \, u\tt,\qquad \vec{x}(0)=\vec{x}_0\\
y\tt&= \vec{c}^T\, \vec{x}\tt
\end{align}
Es ergibt sich als L�sung von $\vec{x}\tt$:
\begin{align}
x\tt = x_0 + \int\limits_0^t \dot{x}(x,t) \d t 
\end{align}
Daf�r numerische L�sung erforderlich. Dabei wird mit der Abtastzeit $T$ gerechnet.

\subsection{Einschrittverfahren}
\subsubsection{Explizites Eulerverfahren}
Einfachstes Verfahren, ergibt sich aus dem Taylorschen Satz erster Ordnung. Steigung wird in jedem Zeitschritt berechnet und als konstant angenommen. (Zur Verdeutlichung im Zuge dieser Arbeit nur eine Zustandsgr��e!)
\begin{align}
		&\frac{\d x}{\d t}=\lambda \dd x + b\dd u \\
		\Rightarrow & \frac{\Delta x}{T} = \lambda \dd x + b \dd u \\
		\Rightarrow & x_{k+1} = x_k+\Delta x_k = x_k + (\lambda \dd x_k + b\dd u_k) \dd T\\
		\Rightarrow & \text{Stabilit�tsanalyse mit digitaler Regelung Methoden machen!!}
\end{align}

\subsubsection{Impliztes Eulerverfahren}
Andere M�glichkeit: Annahme der Steigung des $k+1$-ten Schrittes f�r $\Delta x$.
\begin{align}
	x_{k+1}&= x_k + \Delta x_{k+1}\\
	x_{k+1}&=x_k + \Delta x_{k+1} \ \text{mit}\\
	\Delta x_{k+1} &= (\lambda \dd x_{k+1} + b \dd u_{k+1}) \dd T \ \text{(vorherige Gl. einsetzen)}\\
	\Rightarrow \Delta x_{k+1}&= (1-\lambda \dd T)^{-1} \dd (b\dd u_{k+1} + \lambda x_k)\dd T
\end{align}
Es ergibt sich eine implizite Gleichung f�r die Steigung, und $u_{k+1}$ erforderlich.
Besonders: F�hrt immer zu Stabilit�t (unabh�ngig von der Schrittweite), wenn System stabil! (Eigenwerte anschauen!) ABER: oft auch f�r instabile Systeme stabil!

\subsubsection{Trapezregel}Referenz auf SIEMENS SKRIPT
Kombination der beiden Euler-Verfahren $\Rightarrow$ h�here Genauigkeit!\\
Mittelwert der Steigungen im $k$-ten und $k+1$-ten Schritt:
\begin{align}
	x_{k+1} &= x_k + \Delta x_k = \\
			& = x_k + T \dd \eh \dd \Big( f(x(t+T),t+T) + f(x\tt,t) \Big)
\end{align}
ebenfalls implizites Verfahren!
Aufl�sen oder Referenz auf Extrapolationsverfahren... 

\subsubsection{Runge-Kutta-Verfahren}
Explizites Verfahren
Ordnung, Heun etc.



\subsection{explizite Mehrschrittverfahren}

\subsection{implizite Mehrschrittverfahren}

\subsection{Pr�diktor-Korrektor-Verfahren}

\subsection{Adaptive Verfahren}
\subsection{Extrapolationsverfahren}

% Nach Friedrich: Numerische Methoden:
%\section{Beschreibung der Integrationsverfahren}
%	\subsection{Explizite Einschrittverfahren}
%		\subsubsection{Explizites Eulerverfahren}
%		\subsubsection{Trapezregel}
%		\subsubsection{Runge-Kutta-Verfahren}
%	\subsection{Implizite Einschrittverfahren}
%	\subsection{Impliztes Eulerverfahren}
%	\subsection{explizite Mehrschrittverfahren}
%	\subsection{implizite Mehrschrittverfahren}
%	\subsection{Pr�diktor-Korrektor-Verfahren}

% Nach Faires: Numerische Methoden>
% \subsection{Eulersches Verfahren}
% \subsection{Runge-Kutta Verfahren}
% \subsubsection{Mittelpunktmethode}
% \subsubsection{modifiziertes Eulersches Verfahren $\hat{=}$ implizites Eulerverfahren}
% \subsubsection{Heunsches Verfahren}
% \subsubsection{Runge-Kutta Verfahren 4. Ordnung}
% \subsection{(Pr�diktor-Korrektor Verfahren )$\hat{=}$Mehrschrittverfahren}
% --> Unterscheidung von explizit und implizit (????)
% \subsubsection{Adam-Bashforth ...(explizit)}
% \subsubsection{Adams-Moulton ....(implizit)}
% \subsubsection{Pr�diktor-Korrektor Verfahren: Milnesche, Simpsonsche}
% Ausblick:(?) Extrapolationsverfahren und Adaptive Verfahren


\section{Stabilit�tsanalyse der Methoden}


\section{Anwendung der Integrationsmethoden in Simulationssoftware}
\subsection{PSS Nettomac}
\subsection{PowerFactory}
RMS Simulation Algorithms \\
\textbullet     Highly accurate, fixed or variable step-size integration technique for solving AC and DC network load 
flow and dynamic model equations. This is combined with a non-linear electromechanical model 
representation to enable a high degree of solution accuracy, algorithmic stability and time range validity.\\
\textbullet     A-stable simulation algorithm for the efficient handling of stiff systems. This is applicable to all or any 
individually selected model featuring error-controlled automatic step-size adaptation, ranging from 
milliseconds up to minutes or even hours, including precise handling of interrupts and discontinuities.  
EMT Simulation Algorithms \\
\textbullet     The calculation of initial conditions is carried out prior to the EMT simulation, and is based on a solved 
load flow (symmetrical or asymmetrical). Consequently, there is no need for saving steady state 
conditions being reached after transients are damped out aiming in simulation re-starting under steady 
state conditions.  \\
\textbullet     Special numerical integration methods have been implemented in DIgSILENT PowerFactory in order to 
avoid numerical oscillations caused by switching devices and other non-linear characteristics.  \\
\textbullet    Highly accurate, fixed or variable step-size integration technique for solving AC and DC network 
transients and dynamic model equations. This is combined with a non-linear electromechanical model 
representation to enable a high degree of solution accuracy, algorithmic stability and time range validity. 
\subsection{PSSE}
\subsection{Eurostag}
The advanced dynamic functions of EUROSTAG� allow for the full range of transient, mid and long-term stability to be covered thanks to a robust algorithm using an auto-adaptative integration stepsize.
























\newpage
Hier ist das eigentliche Thema zu bearbeiten.

%\section{Einbindung von Bildern}
%\label{kap:einbindungbilder}
%Abbildungen sind mit Hilfe des Pakets \textit{graphicx} einzuf�gen. Sie k�nnen im PDF-Format durch die Nutzung des folgenden Codes implementiert werden.
%\begin{verbatim}
%\begin{figure}[!htb]\centering
% \includegraphics*[width = \textwidth]{beispiel}
% \caption{Beispiel f�r die Einbindung eines Bildes}
% \label{abb:beispiel}
%\end{figure}
%\end{verbatim}
%Das Ergebnis ist die Anzeige des Bildes, mittig, wie der Text breit mit der angegebenen Unterschrift. Alternativ kann bei der Breite eine absolute Angabe in mm erfolgen. �ber das label \textit{abb:beispiel} kann das Bild referenziert werden.
%\begin{figure}[!htb]\centering
% \includegraphics*[width = \textwidth]{beispiel}
% \caption{Beispiel f�r die Einbindung eines Bildes}
% \label{abb:beispiel}
%\end{figure}
%Um auf das Bild \ref{abb:beispiel} zu verweisen, bedient man sich der folgenden Funktion:
%\begin{verbatim}
%\ref{abb:beispiel}
%\end{verbatim}
%Die referenzierte Nummerierung erfolgt Kapitelweise. Will man weiterhin eine Quelle in der Bildunterschrift angeben, so ist darauf zu achten, dass die Einbindung der Bildunterschriften durch ein optionales Element (eingeschlossen in eckigen Klammern) erweitert wird, welches die Beschriftung f�r das Abbildungsverzeichnis enth�lt:
%\begin{verbatim}
% \caption[Beispiel...]{Beispiel... , aus \cite{schwab}}
%\end{verbatim}
%Diese Variante verhindert, dass LaTex die Quellen bereits im Abbildungsverzeichnis zu z�hlen anf�ngt.
%
%\section{Einbindung von Tabellen}
%\label{kap:einbindungtabellen}
%Die Tabellen sollen mit Hilfe des Pakets \textit{tabularx} eingebunden werden. Im Folgenden ist ein Beispiel f�r die Einbindung von Tabellen aufgef�hrt. Mit
%\begin{verbatim}
%\begin{table}[!htb]
%\centering
%\caption{Beispiel einer Tabelle}
%\label{tab:tabelle1}
%\begin{tabularx}{\textwidth}{|X|c|c|c|c|c|c|c|c|}
%\hline
%        & Spalte 1 & Spalte 2 & Spalte 3 & Spalte 4 & Spalte 5 \\
%\hline
%Zeile 1 &          &          &          &          &          \\
%\hline
%Zeile 2 &          &          &          &          &          \\
%\hline
%Zeile 3 &          &          &          &          &          \\
%\hline
%\end{tabularx}
%\end{table}
%\end{verbatim}
%ergibt sich die folgende Tabellenausgabe \ref{tab:tabelle1}.
%\begin{table}[!htb]
%\centering
%\caption{Beispiel einer Tabelle}
%\label{tab:tabelle1}
%\begin{tabularx}{\textwidth}{|X|c|c|c|c|c|c|c|c|}
%\hline
% 				& Spalte 1 	& Spalte 2 	& Spalte 3 	& Spalte 4 	& Spalte 5 \\
%\hline
%Zeile 1 & 					& 					& 					& 					& \\
%\hline
%Zeile 2 & 					& 					& 					& 					& \\
%\hline
%Zeile 3 & 					& 					& 					& 					& \\
%\hline
%\end{tabularx}
%\end{table}
%Bei Angabe von Quellen in der Tabellen�berschrift ist �hnlich wie im Kapitel \ref{kap:einbindungbilder} zu verfahren.
%
%\section{Eingabe von Gleichungen}
%\label{kap:einbindunggleichungen}
%Die Eingabe von Gleichungen erfolgt nach dem folgenden Beispiel:
%\begin{verbatim}
%\begin{align}
%U=R \cdot I
%\end{align}
%\end{verbatim}
%Das Ergebnis ist die folgende Darstellung mit der automatischen Nummerierung:
%\begin{align}
%U=R \cdot I
%\end{align}
%Durch die Einbindung des eigens definierten Pakets \textit{befehle.sty} stehen zudem die folgenden Funktionen f�r die Gleichungseingabe zur Verf�gung:
%\begin{tabbing}
%XXXXXX 	\= XXXXXXXXXXXXXXXXXXXXXXXXXXXXXXXXXXXXXXXXXXXXXXXXXXXXXXXX\kill
%einh{} 	\> Einheiten\\
%iz{}		\> Nicht-kursiver Index, unten\\
%izo{}		\> Nicht-kursiver Index, oben\\
%izf{}		\> Nicht-kursiver Index, unten, kleinere Schrift\\
%izof{}	\> Nicht-kursiver Index, oben, kleinere Schrift
%\end{tabbing}
%Die Anwendung der Funktionen kann anhand des folgenden Beispiels nachvollzogen werden. Mit dem Code
%\begin{verbatim}
%U\iz{R}=I \cdot R = 10\einh{A} \cdot 1\einh{$\Omega$}=10\einh{V}
%\end{verbatim}
%ergibt sich die folgende Ausgabe:
%\begin{align}
%U\iz{R}=I \cdot R = 10\einh{A} \cdot 1\einh{$\Omega$}=10\einh{V}
%\end{align}
%Weitere Funktionen k�nnen der Datei entnommen und individuell erweitert werden.
%
%\section{Quellenangabe}
%\label{kap:quellenangabe}
%Die Angabe von Quellen ist mit folgendem Code als Beispiel m�glich:
%\begin{verbatim}
%\cite{schwab}
%\end{verbatim}
%Die Ausgabe wird an der Stelle eingef�gt, an der man es einsetzt. Zum Beispiel hier: \cite{schwab}. Die Zahlen werden nach dem Vorkommen im Text vergeben und durchnummeriert. D.h. die n�chste Angabe erh�lt die Nummer zwei und zwar hier: \cite{oedingoswald}. Die Eintr�ge in der Literaturdatenbank k�nnen manuell in der Datei \textit{literatur.bib} oder mit der Software \textit{JabRef} angepasst werden.