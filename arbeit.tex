% EES Latex-Vorlage M�rz 2012
\documentclass[
12pt,               																	% Schriftgr��e
a4paper,            																	% Layout f�r DINA4
german,             																	% deutsche Sprache, global
twoside,            																	% Layout f�r beidseitigen Druck
headinclude,        																	% Kopfzeile wird Seiten-Layouts mit ber�cksichtigt
headsepline,        																	% horizontale Linie unter Kolumnentitel
plainheadsepline,																			% horizontale Linie unter Kolumnentitel auch bei Chapter
BCOR20mm,           																	% Korrektur f�r die Bindung
DIV18,              																	% DIV-Wert f�r die Erstellung des Satzspiegels, siehe scrguide
parskip=half,       																	% Absatzabstand statt Absatzeinzug
openany,            																	% Kapitel k�nnen auf geraden und ungeraden Seiten beginnen
bibliography=totoc,version=first, 										% Literaturverz. wird ins Inhaltsverzeichnis eingetragen
numbers=noenddot,   																	% Kapitelnummern immer ohne Punkt
captions=tableheading,version=first, 									% korrekte Abst�nde bei Tabellen�berschriften
fleqn,             																		% fleqn: Glgen links (statt mittig)
listof=totoc,version=first														% Abbildungs- und Tabellenverzeichnis ins Inhaltsverzeichnis
]{scrbook}


%--------------- Packages ----------------
\usepackage[ngerman]{babel} 													% Neue deutsche Trennmuster
\usepackage[latin1]{inputenc} 												% direkte Eingabe von Umlauten & Co.
\usepackage[T1]{fontenc} 															% T1-Schriften
\usepackage{ae}               												% f�r PDF-Erstellung
\usepackage[format=hang,justification=raggedright,singlelinecheck=off,labelfont=bf,font=small,skip=4pt]{caption} % Captions ausrichten
\usepackage[centertags]{amsmath} 											% AMS-Mathematik, centertags zentriert Nummer bei split
\usepackage{tabularx}																	% erweiterte Tabellen
\usepackage{graphicx}            											% zum Einbinden von Grafiken
\usepackage{float}               											% u.a. genaue Plazierung von Gleitobjekten mit H
\usepackage{setspace}            											% Zeilenabstand einstellbar
\usepackage{scrpage2}           											% Kopf- und Fu�zeilen-Layout 
\usepackage{befehle}  																% eigene Befehle in der Datei definiert
\usepackage{listings}																	% Auflistungen
\usepackage{scrhack}																	% Damit kriegt man den Fehler mit den Kopf- und Fu�zeilen weg
\usepackage{makeidx}																	% Paket f�r das Stichwortverzeichnis
\usepackage{xcolor}																		% Paket f�r die Farben
\usepackage[colorlinks,pdfpagelabels,pdfstartview = FitH,bookmarksopen = true,bookmarksnumbered = true,linkcolor = black,plainpages = false,hypertexnames = false,citecolor = black] {hyperref}																			% Paket f�r das verlinkte PDF
\usepackage[absolute]{textpos}
\usepackage[clearempty]{titlesec}				

%%
\usepackage{tikz}
\usepackage{calc}
\usepackage{setspace}
\usepackage{xparse}
\usetikzlibrary{intersections,decorations.pathreplacing,fit,calc,positioning}

%--------------- Sonstiges ----------------

\pagestyle{scrheadings}																% Kopf- und Fu�zeile...
\renewcommand{\headfont}{\normalfont\sffamily}    		% Kolumnentitel serifenlos
\renewcommand{\pnumfont}{\normalfont\sffamily}    		% Seitennummern serifenlos
\ihead[]{\headmark}              											% Kopfzeile innen
\ohead[\pagemark]{\pagemark}     											% Kopfzeile au�en
\ifoot[]{} 																						% Fu�zeile innen
\ofoot[]{}    																				% Fu�zeile au�en	
\setlength{\headheight}{1.5\baselineskip}
\onehalfspacing																				% 1,5 Zeilenabstand
%\typearea[current]{current}        									% Neuberechnung des Satzspiegels mit alten Werten nach �nderung von Zeilenabstand,etc
\renewcommand{\bibname}{Literatur und Quellen} 				% Literaturverzeichnisbezeichnung
\renewcommand{\figurename}{Abb.}   										% Abbildungsbezeichnung
\renewcommand{\listfigurename}{Abbildungsverzeichnis} % Abbildungsverzeichnisbezeichnung
\renewcommand{\captionfont}{\small}										% Bildunterschriften klein kursiv
\graphicspath{{figs/}{bilder/}}    										% Bildverzeichnis


%--------------- Code-Darstellung ----------------
\lstdefinelanguage{Matlab2}
{morekeywords={for,if,else,end,function,disp,clc,input,subplot,plot,figure,axis,ylabel,xlabel,legend,set,gca,gcf,grid,switch,case,otherwise,eval,ss,sisotool,linmod,clear,load,addpath,pwd,},
sensitive=false,
morecomment=[l]{//},
morecomment=[l]{\%},
morecomment=[s]{/*}{*/},
morestring=[b]",
}
\definecolor{MyDarkBlue}{rgb}{0,0.08,0.45}
\definecolor{fett}{rgb}{0.965,0.984,0.961}
\definecolor{OliveGreen}{rgb}{0.43,0.66,0.45}
\lstset{language=Matlab2, basicstyle=\tiny, backgroundcolor=\color{fett}, keywordstyle=\color{MyDarkBlue}\bfseries ,numbers=left, numberstyle=\tiny, tabsize=2, commentstyle=\color{OliveGreen}}
\makeindex


%----------------------- Beginn des Dokuments -----------------------
\begin{document}
\pagenumbering{Roman}																	% R�mische Seitennummerierung
\newcommand{\dd}{\cdot}
\renewcommand{\tt}{(t)}
\newcommand{\kk}{[k]}
\newcommand{\zz}{(z)}
\newcommand{\vs}{(s)}
\renewcommand{\u}[1]{u_#1(t)}
\newcommand{\y}[1]{y_#1(t)}
\newcommand{\x}[1]{x_#1(t)}
\newcommand{\itt}{\int_0^t}
\newcommand{\iti}{\int_0^\infty}
\newcommand{\etT}{\e^{-\frac{t}{T}}}
\newcommand{\ud}[1]{\underline{\dot{#1}}}
\newcommand{\ut}[1]{\underline{\tilde{#1}}}
\newcommand{\uh}[1]{\underline{\hat{#1}}}
\newcommand{\eps}{\ul{\varepsilon}}
\newcommand{\xk}{\underline{x}_k}
\newcommand{\ra}{\curvearrowright}
\newcommand{\eh}{\frac{1}{2}}
\newcommand{\ul}[1]{\underline{#1}}
\newcommand{\A}{\underline{A}}
\newcommand{\At}{\underline{\tilde{A}}}
\newcommand{\T}{\underline{T}}
\newcommand{\I}{\underline{I}}
\newcommand{\Pm}{\underline{P}}
\newcommand{\bv}{\underline{b}}
\newcommand{\bvt}{\underline{\tilde{b}}}
\newcommand{\xv}{\underline{x}}
\newcommand{\kt}{\ul{k}^T}
\newcommand{\ex}{\ul{e}_x}
\newcommand{\cv}{\underline{c}^T}
\newcommand{\cvt}{\underline{\tilde{c}}^T}
\newcommand{\ct}{\frac{c}{\Theta_M}}
\renewcommand{\d}{\mathrm{d}}
\newcommand{\jo}{\im\omega}
\newcommand{\ag}{\alpha\gamma}
\newcommand{\bg}{\beta\gamma}
\newcommand{\xvz}{\begin{bmatrix}
x_1\\x_2
\end{bmatrix}}
\newcommand{\xpz}{\begin{bmatrix}
\dot{x}_1\\ \dot{x}_2
\end{bmatrix}}
\newcommand{\PBZ}{\stackrel{\text{PBZ}}{=}}
\newcommand{\adj}{\mathrm{adj}}
%%%%%%%%%%%%%%%%%%%%%%%%%%
%TIKZ:
\tikzset{%
	>=latex,
  highlight/.style={rectangle,rounded corners,draw=orange,very thin,fill opacity=0.5,inner sep=-2pt},
  frame/.style={
	rectangle,minimum size=9mm,thick,draw=black
	},
 sum/.style={
	circle,minimum size =3mm,thick,draw=black, inner sep = 0pt
	},
 knot/.style={
	circle,minimum size =2mm,fill=black, inner sep=0pt
	},
 skip  loop/.style={to  path={--  ++(0,#1)  -|  (\tikztotarget)}},
 nonlin/.style = {frame, double,double distance=1pt},
 times/.style = {alias=sourcenode,frame, 
        append after command={
        ($(sourcenode.center)+(-0.2,0.2)$)--($(sourcenode.center)+(0.2,-0.2)$)
	  ($(sourcenode.center)+(-0.2,-0.2)$)--($(sourcenode.center)+(0.2,0.2)$)	
        }
	}
}



\newcommand{\tikzrightbrace}[4][0.5\textwidth]{%
%%% Ein rechter brace der die nodes 2 und 3 umschlie�t.
%%% rechts daneben wird der Text 4 mit der Breite 1 *optional angezeigt!
\tikz[overlay,remember picture]{
\node[fit=(#2.north east) (#3.south east),inner sep =0] (gesamt#2) {};
\draw[decorate, decoration=brace] (gesamt#2.north east)--(gesamt#2.south east);
\path (gesamt#2.east) node[right,xshift=2mm]{\parbox{#1}{#4}};
}
}


\newcommand{\tikzmark}[3][3pt]{\tikz[remember picture,baseline=(#2.base)] \node[inner sep=#1] (#2) {#3};}
%tikzmark: markiert inhalt als tikznode: 1(optional): inner sep der node
%									2: name der Node
%									3: Inhalt der Node
\newcommand{\tikzmarktwo}[3][2pt]{\tikz[remember picture,baseline=(#2.base)] 
\node[inner ysep=#1,inner xsep=0pt, outer xsep=0pt] (#2) {#3};}
%%tikzmark: markiert inhalt als tikznode: 1(optional): inner ysep der node
%%									2(optional): inner xsep der node
%%									3: name der Node
%%									4: Inhalt der Node

%% Befehl, um einfach Beschriftungen mit Pfeil hinzuzuf�gen: (h�lt Abstand nach unten/oben automatisch ein! (standardm��ig in footnotesize aber durch voranstellen einer gr��e �berschreibbar!)
% #1 (optional, def 0.5cm): Abstand der Beschriftung vom zu Beschriftenden
% #2 (optional, def 270): Winkel, in dem die Beschriftung angebracht wird.
% #3 zu beschriftender Text
% #4 Beschriftung 
\DeclareDocumentCommand \pin {O{0.5cm} O{270} m m}{
\tikz[baseline=(mynode.base), pin distance=#1,every pin edge/.style={<-}, inner sep=1pt]{
\node(mynode)[pin=#2:{\footnotesize  #4}]{#3};
\coordinate (pin) at (current bounding box.south);
\path let \p1=(pin), \p2=(mynode.center) in %
	node[inner sep=0pt](da) at (\x2,\y1){} node(gesamt)[fit=(mynode)(da),inner sep =0pt]{};
\pgfresetboundingbox
\useasboundingbox (gesamt.north west)rectangle(gesamt.south east);
%\draw[fill=red] (current bounding box.south west) circle(0.02cm);
%\draw[fill=red] (current bounding box.south east) circle(0.02cm);
}
}

\DeclareDocumentCommand \pintest{O{0.5cm} O{270} m m}{#1 #2 #3 #4}

\newcommand{\Highlight}[1][submatrix]{%
    \tikz[overlay,remember picture]{
    \node[highlight,fit=(left.north west) (right.south east)] (#1) {};}
}
\newcommand{\highlight}[2]{%
	\tikz[overlay,remember picture]{
	\node[highlight,fit=(#1.north west)(#1.south east)] (#2){};
	}
}
\newcommand{\underbrac}[2]{%
	\tikz[overlay,remember picture, very thin, orange]{
	\node[inner sep=-2pt,fit=(#1.north west)(#1.south east)] (#2){};
	\path (#1.south west)++(0.065,0.05) edge($(#1.south west)+(0.065,0)$)
		  ($(#1.south west)+(0.065,0)$) edge ($(#1.south east)-(0.065,0)$)
  	 	 ($(#1.south east)-(0.065,0)$)++(0,0.05) edge  ($(#1.south east)-(0.065,0)$);
	}
}
%%%%%%%%%%%%%%%%%%%%%%%%%%%%%%%%%%%%%%%
\newcommand{\tikzdoubleul}[1]{ % doppelte Unterstreichungen auch fuer groessere Objekte
\tikzmarktwo{tikzdoubleul}{#1}
\tikz[overlay, remember picture]{
\path (tikzdoubleul.south west) edge[double, thick, double distance = 1.5pt] (tikzdoubleul.south east);
}
}

\newcommand{\uldash}[1]{ % gestrichelte Unterstreichung
\tikzmarktwo[2pt]{uldash}{#1}
\tikz[overlay, remember picture]{
\path (uldash.south west) edge[draw, thick, dashed] (uldash.south east);
}
}

\newcommand{\tikzcancel}[1]{ % cancel gestrichelt
\tikzmarktwo{tikzcancel}{#1}
\tikz[overlay, remember picture]{
\path ($(tikzcancel.south west)-(1pt,2pt)$) edge[draw,dotted,thick,color =gray] ($(tikzcancel.north east)+(1pt,2pt)$);
}
}


%%%%%%%%%%%%%%%%%%%%%%%%%%%%%%
% fuer bessere Underbraces!!
% normal den underbrace Befehl verwenden aber dann den Text mathclapen! Bsp: \underbrace{a}{\mathclap{text}}
 \def\mathllap{\mathpalette\mathllapinternal}
 \def\mathllapinternal#1#2{%
 \llap{$\mathsurround=0pt#1{#2}$}%  $
 }
 \def\clap#1{\hbox  to  0pt{\hss#1\hss}}
 \def\mathclap{\mathpalette\mathclapinternal}
 \def\mathclapinternal#1#2{%
 \clap{$\mathsurround=0pt#1{#2}$}%
}
 \def\mathrlap{\mathpalette\mathrlapinternal}
 \def\mathrlapinternal#1#2{%
 \rlap{$\mathsurround=0pt#1{#2}$}%  $
 }
%%%%%%%%%%%%%%%%%%%%%%%%%%%%%%
%%%%%%%%%%%%%%%%%%%%%%%%%%%%
% underbrackets!
\makeatletter
\def\overbracket#1{\mathop{\vbox{\ialign{##\crcr\noalign{\kern3\p@}
\downbracketfill\crcr\noalign{\kern3\p@\nointerlineskip}
$\hfil\displaystyle{#1}\hfil$\crcr}}}\limits}
\def\underbracket#1{\mathop{\vtop{\ialign{##\crcr
$\hfil\displaystyle{#1}\hfil$\crcr\noalign{\kern3\p@\nointerlineskip}
\upbracketfill\crcr\noalign{\kern3\p@}}}}\limits}
\def\overparenthesis#1{\mathop{\vbox{\ialign{##\crcr\noalign{\kern3\p@}
\downparenthfill\crcr\noalign{\kern3\p@\nointerlineskip}
$\hfil\displaystyle{#1}\hfil$\crcr}}}\limits}
\def\underparenthesis#1{\mathop{\vtop{\ialign{##\crcr
$\hfil\displaystyle{#1}\hfil$\crcr\noalign{\kern3\p@\nointerlineskip}
\upparenthfill\crcr\noalign{\kern3\p@}}}}\limits}
\def\downparenthfill{$\m@th\braceld\leaders\vrule\hfill\bracerd$}
\def\upparenthfill{$\m@th\bracelu\leaders\vrule\hfill\braceru$}
\def\upbracketfill{$\m@th\makesm@sh{\llap{\vrule\@height3\p@\@width.7\p@}}%
\leaders\vrule\@height.7\p@\hfill
\makesm@sh{\rlap{\vrule\@height3\p@\@width.7\p@}}$}
\def\downbracketfill{$\m@th
\makesm@sh{\llap{\vrule\@height.7\p@\@depth2.3\p@\@width.7\p@}}%
\leaders\vrule\@height.7\p@\hfill
\makesm@sh{\rlap{\vrule\@height.7\p@\@depth2.3\p@\@width.7\p@}}$}
\makeatother
%%%%%%%%%%%%%%%%%%%%%%%%%%%%%%%%%%%%

% Markierung als Studenten rechnen lassen>
\newcommand{\markstudent}[3] %Argumente: 1: Abstand zwischen jetziger Zeile und Inhalt (Bei Equations: Baselineskip)
% 2.: L�nge des Inhalts /Markierung
%3. : Inhalt
{
\ifLeiter
\begin{minipage}[t]{0.05\textwidth}	
	\vspace{#1}
	\unitlength1cm
	{\color{orange}
	\makebox[0.025\textwidth]{\line(0,-1){#2}} 
	\makebox[0.025\textwidth]{\line(0,-1){#2}}}
\end{minipage}
\begin{minipage}[t]{0.95\textwidth}
#3
\end{minipage}
\else
#3
\fi
}

% Markierung f�r �bungsleiter>
\newcommand{\markteacher}[3] %Argumente: 1: Abstand zwischen jetziger Zeile und Inhalt (Bei Equations: Baselineskip)
% 2.: L�nge des Inhalts /Markierung
%3. : Inhalt
{
\ifLeiter
\begin{minipage}[t]{0.05\textwidth}	
	\vspace{#1}
	\unitlength1cm
	{\color{ForestGreen}
	\makebox[0.025\textwidth]{\begin{sideways}\uwave{\hspace{#2}}\end{sideways}}
	\makebox[0.025\textwidth]{\begin{sideways}\uwave{\hspace{#2}}\end{sideways}}  
	}
\end{minipage}
\begin{minipage}[t]{0.95\textwidth}
#3
\end{minipage}
\else
#3
\fi
}

% Markierung f�r �bungsleiter, einzeilig!
\newcommand{\markteachers}[1] 
{
\ifLeiter
\begin{minipage}[t]{0.05\textwidth}	
	\vspace{-0.75\baselineskip}
	\unitlength1cm
	{\color{OliveGreen}
	\makebox[0.025\textwidth]{\begin{sideways}\uwave{\hspace{0.85\baselineskip}}\end{sideways}}
	\makebox[0.025\textwidth]{\begin{sideways}\uwave{\hspace{0.85\baselineskip}}\end{sideways}}  
	}
\end{minipage}
\hspace{-0.5cm}
\begin{minipage}[t]{0.95\textwidth}
#1%
\end{minipage}
\else
#1
\fi
}

% Abs�tze machen:
\newcommand{\absatz}[2]
{
\ifdefined\wortlang 
\else
\newlength{\wortlang}
\fi
\settowidth{\wortlang}{#1}
\ifdim \wortlang<0.5\linewidth
\begin{tabular}{p{\wortlang}p{\linewidth-4\tabcolsep-\wortlang}}
\else
\begin{tabular}{p{0.5\linewidth-2\tabcolsep}p{0.5\linewidth-2\tabcolsep}}
\fi
#1 & #2
\end{tabular}
\let\wortlang\relax
}




\newcommand{\koords}{
\put(0,0){\makebox(0,0)[rb]{
\begin{picture}(0,0)
\put(4,0){\line(0,0){0,1}$4$}
\put(3,0){\line(0,0){0,1}$3$}
\put(2,0){\line(0,0){0,1}$2$}
\put(1,0){\line(0,0){0,1}$1$}
\put(0,0){\line(0,0){0,1}$0$}
\put(-1,0){\line(0,0){0,1}$1$}
\put(-2,0){\line(0,0){0,1}$2$}
\put(-3,0){\line(0,0){0,1}$3$}
\put(-4,0){\line(0,0){0,1}$4$}
\put(-5,0){\line(0,0){0,1}$5$}
\put(-6,0){\line(0,0){0,1}$6$}
\put(-7,0){\line(0,0){0,1}$7$}
\put(-8,0){\line(0,0){0,1}$8$}
\put(-9,0){\line(0,0){0,1}$9$}
\put(-10,0){\line(0,0){0,1}$10$}
\put(-11,0){\line(0,0){0,1}$11$}
\put(-12,0){\line(0,0){0,1}$12$}
\put(-13,0){\line(0,0){0,1}$13$}
\put(-14,0){\line(0,0){0,1}$14$}
\put(-15,0){\line(0,0){0,1}$15$}
\put(-16,0){\line(0,0){0,1}$16$}
\put(-17,0){\line(0,0){0,1}$17$}
\put(-18,0){\line(0,0){0,1}$18$}
\put(-19,0){\line(0,0){0,1}$19$}
\put(-20,0){\line(0,0){0,1}$20$}
\put(-21,0){\line(0,0){0,1}$21$}
\put(-22,0){\line(0,0){0,1}$22$}
\put(-23,0){\line(0,0){0,1}$23$}
\put(-24,0){\line(0,0){0,1}$24$}
\put(0,-5){\line(-1,0){0,1}$5$}
\put(0,-4){\line(-1,0){0,1}$4$}
\put(0,-3){\line(-1,0){0,1}$3$}
\put(0,-2){\line(-1,0){0,1}$2$}
\put(0,-1){\line(-1,0){0,1}$1$}
\put(0,0){\line(-1,0){0,1}}
\put(0,1){\line(-1,0){0,1}$1$}
\put(0,2){\line(-1,0){0,1}$2$}
\put(0,3){\line(-1,0){0,1}$3$}
\put(0,4){\line(-1,0){0,1}$4$}
\put(0,5){\line(-1,0){0,1}$5$}
\put(0,6){\line(-1,0){0,1}$6$}
\put(0,7){\line(-1,0){0,1}$7$}
\put(0,8){\line(-1,0){0,1}$8$}
\put(0,9){\line(-1,0){0,1}$9$}
\put(0,10){\line(-1,0){0,1}$10$}
\put(0,11){\line(-1,0){0,1}$11$}
\put(0,12){\line(-1,0){0,1}$12$}
\put(0,13){\line(-1,0){0,1}$13$}
\put(0,14){\line(-1,0){0,1}$14$}
\put(0,15){\line(-1,0){0,1}$15$}
\put(0,16){\line(-1,0){0,1}$16$}
\end{picture}
}
}
}

\renewcommand{\L}{\mathrsfs{L}}

\thispagestyle{empty}
\setcounter{page}{-1}

\begin{textblock*}{47mm}(30mm,33mm)
\includegraphics[width=47mm]{ees}
\end{textblock*}

\begin{textblock*}{93mm}(127mm,192mm)
\includegraphics[width=93mm]{fausiegel}
\end{textblock*}

{\raggedleft
\textsc{Lehrstuhl f�r Elektrische Energiesysteme}\\
Vorstand: Univ.-Prof. Dr.-Ing. Matthias Luther
\par}

\vspace{51mm}

{\centering

%##########################################################
% BEARBEITUNGSFELDER
\large{Seminararbeit} \\	
\Large{Integrationsverfahren zur Simulation im Zeitbereich}
%##########################################################

\par}

\vspace{105mm}

{\raggedright
\begin{tabbing}
XX \= XXXXXXXXXXX \= XXXXXXXXXXXXXXXXXXXXXX \kill
%##########################################################
% BEARBEITUNGSFELDER
		\> \textbf{Bearbeiter:} 	\> Simon Kerschbaum \\
		\>												\> 21445549		\\
 		\>												\>									\\
		\> \textbf{Betreuer:}			\> Anatoli Semerow \\
		\>												\>									\\
		\> \textbf{Abgabedatum:}	\> 13.12.2013															
%##########################################################
\end{tabbing}
\par}

\thispagestyle{empty}
\cleardoublepage														% Deckblatt

\include{inhalt/erklaerung}														% Erkl�rung der Eigenarbeit

\chapter*{Aufgabenstellung der Arbeit}
%\vspace*{4cm}

{\large \textbf{Thema:} \parbox[t]{0.8\textwidth}{Thema der Arbeit}}
\\

Hier wird die Aufgabenstellung beschrieben. Die Notwendigkeit dieser h�ngt vom Betreuer ab. 															% Aufgabenstellung (optional)

%\include{inhalt/vorwort}															% Vorwort (optional)

\cleardoublepage
\begin{spacing}{1.15}																	% evtl. kleinerer Zeilenabstand im IV, AV, TV
\pdfbookmark[1]{Inhaltsverzeichnis}{toc}							% Inhaltsverzeichnis bei den Lesezeichen rein
\tableofcontents 																			% Inhaltsverzeichnis erzeugen
\listoffigures   																			% Abbildungsverzeichnis
\listoftables   																			% Tabellenverzeichnis
\end{spacing}

%\include{inhalt/abstract}															% Abstract

\mainmatter																						% Hauptteil beginnt

\chapter{Einleitung}
\label{kap:einleitung}
Die Analyse von komplexen Energieversorgungsnetzen ist ohne das Hilfsmittel Computer kaum denkbar. Die Gr��e der Netze sowie die Anzahl deren Komponenten und Speicherelemente macht eine rein analytische Berechnung sehr schwierig. Der Einsatz des Computers hat es dem Ingenieur auch in der Energietechnik erm�glicht, verschiedenste dynamische Systeme zu simulieren, um dadurch Systemeigenschaften zu beobachten oder mit unerprobten Konzepten zu experimentieren. 
Dynamische Systeme sind dabei im allgemeinen dadurch gekennzeichnet, dass sie Energiespeicher enthalten, die zu zeitver�nderlichen Vorg�ngen f�hren. Aus physikalischer Sicht sind diese Vorg�nge durch Differentialgleichungen charakterisiert, deren L�sungen sich nicht immer in analytischer Form oder zumindest einfach angeben lassen.
Der Computer liefert die M�glichkeit, mit numerischen Methoden N�herungen dieser L�sungen zu berechnen.
Hierf�r gibt es eine Vielzahl verschiedener Methoden, welche unter gewissen Voraussetzungen zu unterschiedlichen Ergebnissen oder zu abweichender Genauigkeit f�hren k�nnen. 
Um Simulationssoftware anwenden zu k�nnen ist es deshalb notwendig, dass der Benutzer die Grundlagen der hinter der Software steckenden Numerik versteht.

Aus diesem Grund werden in dieser Arbeit die wichtigsten Verfahren zur numerischen Simulation von dynamischen Systemen vorgestellt. Dabei wird auf Einschrittverfahren, sowie auf Mehrschrittverfahren eingegangen. Dar�ber hinaus wird die Vorgehensweise der Schrittweitensteuerung erl�utert.
Im \ref{sec:stabi}. Kapitel werden die verschiedenen Verfahren auf Stabilit�t untersucht, welche eine essenzielle Eigenschaft der numerischen Verfahren darstellt.
Um die Bedeutung der Verfahren zu verdeutlichen, werden in Kapitel \ref{sec:anwend} einige Softwarepakete zur Netzsimulation auf ihre verwendeten Verfahren untersucht.
														% Einleitung (Bezeichnung passend zur Arbeit!)

\chapter{Hauptteil}
\label{kap:hauptteil}
\section{Beschreibung der Integrationsverfahren}
Ausgangslage: Numerische L�sung von Anfangswertproblemen. Systeme beschrieben durch allgemeine Zustandsdarstellung:
(Zur Einfachheit Eingr��ensysteme.)
\begin{align}
\dot{\vec{x}}\tt&= \vec{A}\, \vec{x}\tt + \vec{b} \, u\tt,\qquad \vec{x}(0)=\vec{x}_0\\
y\tt&= \vec{c}^T\, \vec{x}\tt
\end{align}
Es ergibt sich als L�sung von $\vec{x}\tt$:
\begin{align}
x\tt = x_0 + \int\limits_0^t \dot{x}(x,t) \d t 
\end{align}
Daf�r numerische L�sung erforderlich. Dabei wird mit der Abtastzeit $T$ gerechnet.

\subsection{Einschrittverfahren}
\subsubsection{Explizites Eulerverfahren}
Einfachstes Verfahren, ergibt sich aus dem Taylorschen Satz erster Ordnung. Steigung wird in jedem Zeitschritt berechnet und als konstant angenommen. (Zur Verdeutlichung im Zuge dieser Arbeit nur eine Zustandsgr��e!)
\begin{align}
		&\frac{\d x}{\d t}=\lambda \dd x + b\dd u \\
		\Rightarrow & \frac{\Delta x}{T} = \lambda \dd x + b \dd u \\
		\Rightarrow & x_{k+1} = x_k+\Delta x_k = x_k + (\lambda \dd x_k + b\dd u_k) \dd T\\
		\Rightarrow & \text{Stabilit�tsanalyse mit digitaler Regelung Methoden machen!!}
\end{align}

\subsubsection{Impliztes Eulerverfahren}
Andere M�glichkeit: Annahme der Steigung des $k+1$-ten Schrittes f�r $\Delta x$.
\begin{align}
	x_{k+1}&= x_k + \Delta x_{k+1}\\
	x_{k+1}&=x_k + \Delta x_{k+1} \ \text{mit}\\
	\Delta x_{k+1} &= (\lambda \dd x_{k+1} + b \dd u_{k+1}) \dd T \ \text{(vorherige Gl. einsetzen)}\\
	\Rightarrow \Delta x_{k+1}&= (1-\lambda \dd T)^{-1} \dd (b\dd u_{k+1} + \lambda x_k)\dd T
\end{align}
Es ergibt sich eine implizite Gleichung f�r die Steigung, und $u_{k+1}$ erforderlich.
Besonders: F�hrt immer zu Stabilit�t (unabh�ngig von der Schrittweite), wenn System stabil! (Eigenwerte anschauen!) ABER: oft auch f�r instabile Systeme stabil!

\subsubsection{Trapezregel}Referenz auf SIEMENS SKRIPT
Kombination der beiden Euler-Verfahren $\Rightarrow$ h�here Genauigkeit!\\
Mittelwert der Steigungen im $k$-ten und $k+1$-ten Schritt:
\begin{align}
	x_{k+1} &= x_k + \Delta x_k = \\
			& = x_k + T \dd \eh \dd \Big( f(x(t+T),t+T) + f(x\tt,t) \Big)
\end{align}
ebenfalls implizites Verfahren!
Aufl�sen oder Referenz auf Extrapolationsverfahren... 

\subsubsection{Runge-Kutta-Verfahren}
Explizites Verfahren
Ordnung, Heun etc.



\subsection{explizite Mehrschrittverfahren}

\subsection{implizite Mehrschrittverfahren}

\subsection{Pr�diktor-Korrektor-Verfahren}

\subsection{Adaptive Verfahren}
\subsection{Extrapolationsverfahren}

% Nach Friedrich: Numerische Methoden:
%\section{Beschreibung der Integrationsverfahren}
%	\subsection{Explizite Einschrittverfahren}
%		\subsubsection{Explizites Eulerverfahren}
%		\subsubsection{Trapezregel}
%		\subsubsection{Runge-Kutta-Verfahren}
%	\subsection{Implizite Einschrittverfahren}
%	\subsection{Impliztes Eulerverfahren}
%	\subsection{explizite Mehrschrittverfahren}
%	\subsection{implizite Mehrschrittverfahren}
%	\subsection{Pr�diktor-Korrektor-Verfahren}

% Nach Faires: Numerische Methoden>
% \subsection{Eulersches Verfahren}
% \subsection{Runge-Kutta Verfahren}
% \subsubsection{Mittelpunktmethode}
% \subsubsection{modifiziertes Eulersches Verfahren $\hat{=}$ implizites Eulerverfahren}
% \subsubsection{Heunsches Verfahren}
% \subsubsection{Runge-Kutta Verfahren 4. Ordnung}
% \subsection{(Pr�diktor-Korrektor Verfahren )$\hat{=}$Mehrschrittverfahren}
% --> Unterscheidung von explizit und implizit (????)
% \subsubsection{Adam-Bashforth ...(explizit)}
% \subsubsection{Adams-Moulton ....(implizit)}
% \subsubsection{Pr�diktor-Korrektor Verfahren: Milnesche, Simpsonsche}
% Ausblick:(?) Extrapolationsverfahren und Adaptive Verfahren


\section{Stabilit�tsanalyse der Methoden}


\section{Anwendung der Integrationsmethoden in Simulationssoftware}
\subsection{PSS Nettomac}
\subsection{PowerFactory}
RMS Simulation Algorithms \\
\textbullet     Highly accurate, fixed or variable step-size integration technique for solving AC and DC network load 
flow and dynamic model equations. This is combined with a non-linear electromechanical model 
representation to enable a high degree of solution accuracy, algorithmic stability and time range validity.\\
\textbullet     A-stable simulation algorithm for the efficient handling of stiff systems. This is applicable to all or any 
individually selected model featuring error-controlled automatic step-size adaptation, ranging from 
milliseconds up to minutes or even hours, including precise handling of interrupts and discontinuities.  
EMT Simulation Algorithms \\
\textbullet     The calculation of initial conditions is carried out prior to the EMT simulation, and is based on a solved 
load flow (symmetrical or asymmetrical). Consequently, there is no need for saving steady state 
conditions being reached after transients are damped out aiming in simulation re-starting under steady 
state conditions.  \\
\textbullet     Special numerical integration methods have been implemented in DIgSILENT PowerFactory in order to 
avoid numerical oscillations caused by switching devices and other non-linear characteristics.  \\
\textbullet    Highly accurate, fixed or variable step-size integration technique for solving AC and DC network 
transients and dynamic model equations. This is combined with a non-linear electromechanical model 
representation to enable a high degree of solution accuracy, algorithmic stability and time range validity. 
\subsection{PSSE}
\subsection{Eurostag}
The advanced dynamic functions of EUROSTAG� allow for the full range of transient, mid and long-term stability to be covered thanks to a robust algorithm using an auto-adaptative integration stepsize.
























\newpage
Hier ist das eigentliche Thema zu bearbeiten.

%\section{Einbindung von Bildern}
%\label{kap:einbindungbilder}
%Abbildungen sind mit Hilfe des Pakets \textit{graphicx} einzuf�gen. Sie k�nnen im PDF-Format durch die Nutzung des folgenden Codes implementiert werden.
%\begin{verbatim}
%\begin{figure}[!htb]\centering
% \includegraphics*[width = \textwidth]{beispiel}
% \caption{Beispiel f�r die Einbindung eines Bildes}
% \label{abb:beispiel}
%\end{figure}
%\end{verbatim}
%Das Ergebnis ist die Anzeige des Bildes, mittig, wie der Text breit mit der angegebenen Unterschrift. Alternativ kann bei der Breite eine absolute Angabe in mm erfolgen. �ber das label \textit{abb:beispiel} kann das Bild referenziert werden.
%\begin{figure}[!htb]\centering
% \includegraphics*[width = \textwidth]{beispiel}
% \caption{Beispiel f�r die Einbindung eines Bildes}
% \label{abb:beispiel}
%\end{figure}
%Um auf das Bild \ref{abb:beispiel} zu verweisen, bedient man sich der folgenden Funktion:
%\begin{verbatim}
%\ref{abb:beispiel}
%\end{verbatim}
%Die referenzierte Nummerierung erfolgt Kapitelweise. Will man weiterhin eine Quelle in der Bildunterschrift angeben, so ist darauf zu achten, dass die Einbindung der Bildunterschriften durch ein optionales Element (eingeschlossen in eckigen Klammern) erweitert wird, welches die Beschriftung f�r das Abbildungsverzeichnis enth�lt:
%\begin{verbatim}
% \caption[Beispiel...]{Beispiel... , aus \cite{schwab}}
%\end{verbatim}
%Diese Variante verhindert, dass LaTex die Quellen bereits im Abbildungsverzeichnis zu z�hlen anf�ngt.
%
%\section{Einbindung von Tabellen}
%\label{kap:einbindungtabellen}
%Die Tabellen sollen mit Hilfe des Pakets \textit{tabularx} eingebunden werden. Im Folgenden ist ein Beispiel f�r die Einbindung von Tabellen aufgef�hrt. Mit
%\begin{verbatim}
%\begin{table}[!htb]
%\centering
%\caption{Beispiel einer Tabelle}
%\label{tab:tabelle1}
%\begin{tabularx}{\textwidth}{|X|c|c|c|c|c|c|c|c|}
%\hline
%        & Spalte 1 & Spalte 2 & Spalte 3 & Spalte 4 & Spalte 5 \\
%\hline
%Zeile 1 &          &          &          &          &          \\
%\hline
%Zeile 2 &          &          &          &          &          \\
%\hline
%Zeile 3 &          &          &          &          &          \\
%\hline
%\end{tabularx}
%\end{table}
%\end{verbatim}
%ergibt sich die folgende Tabellenausgabe \ref{tab:tabelle1}.
%\begin{table}[!htb]
%\centering
%\caption{Beispiel einer Tabelle}
%\label{tab:tabelle1}
%\begin{tabularx}{\textwidth}{|X|c|c|c|c|c|c|c|c|}
%\hline
% 				& Spalte 1 	& Spalte 2 	& Spalte 3 	& Spalte 4 	& Spalte 5 \\
%\hline
%Zeile 1 & 					& 					& 					& 					& \\
%\hline
%Zeile 2 & 					& 					& 					& 					& \\
%\hline
%Zeile 3 & 					& 					& 					& 					& \\
%\hline
%\end{tabularx}
%\end{table}
%Bei Angabe von Quellen in der Tabellen�berschrift ist �hnlich wie im Kapitel \ref{kap:einbindungbilder} zu verfahren.
%
%\section{Eingabe von Gleichungen}
%\label{kap:einbindunggleichungen}
%Die Eingabe von Gleichungen erfolgt nach dem folgenden Beispiel:
%\begin{verbatim}
%\begin{align}
%U=R \cdot I
%\end{align}
%\end{verbatim}
%Das Ergebnis ist die folgende Darstellung mit der automatischen Nummerierung:
%\begin{align}
%U=R \cdot I
%\end{align}
%Durch die Einbindung des eigens definierten Pakets \textit{befehle.sty} stehen zudem die folgenden Funktionen f�r die Gleichungseingabe zur Verf�gung:
%\begin{tabbing}
%XXXXXX 	\= XXXXXXXXXXXXXXXXXXXXXXXXXXXXXXXXXXXXXXXXXXXXXXXXXXXXXXXX\kill
%einh{} 	\> Einheiten\\
%iz{}		\> Nicht-kursiver Index, unten\\
%izo{}		\> Nicht-kursiver Index, oben\\
%izf{}		\> Nicht-kursiver Index, unten, kleinere Schrift\\
%izof{}	\> Nicht-kursiver Index, oben, kleinere Schrift
%\end{tabbing}
%Die Anwendung der Funktionen kann anhand des folgenden Beispiels nachvollzogen werden. Mit dem Code
%\begin{verbatim}
%U\iz{R}=I \cdot R = 10\einh{A} \cdot 1\einh{$\Omega$}=10\einh{V}
%\end{verbatim}
%ergibt sich die folgende Ausgabe:
%\begin{align}
%U\iz{R}=I \cdot R = 10\einh{A} \cdot 1\einh{$\Omega$}=10\einh{V}
%\end{align}
%Weitere Funktionen k�nnen der Datei entnommen und individuell erweitert werden.
%
%\section{Quellenangabe}
%\label{kap:quellenangabe}
%Die Angabe von Quellen ist mit folgendem Code als Beispiel m�glich:
%\begin{verbatim}
%\cite{schwab}
%\end{verbatim}
%Die Ausgabe wird an der Stelle eingef�gt, an der man es einsetzt. Zum Beispiel hier: \cite{schwab}. Die Zahlen werden nach dem Vorkommen im Text vergeben und durchnummeriert. D.h. die n�chste Angabe erh�lt die Nummer zwei und zwar hier: \cite{oedingoswald}. Die Eintr�ge in der Literaturdatenbank k�nnen manuell in der Datei \textit{literatur.bib} oder mit der Software \textit{JabRef} angepasst werden.														% Hauptteil (Bezeichnung passend zur Arbeit!)

\chapter{Zusammenfassung und Ausblick}
\label{kap:zusammenfassung}
In dieser Arbeit wurden die wichtigsten numerischen Integrationsverfahren f�r Anfangswertprobleme kurz vorgestellt. Dabei wurde sowohl auf explizite als auch implizite Verfahren eingegangen und zwischen Einschritt- und Mehrschrittverfahren unterschieden.
Die Vorgehensweise der Schrittweitensteuerung wurde dargelegt und deren Anwendung durch eingebettete Runge-Kutta-Verfahren beschrieben.\\
Dar�ber hinaus konnte eine allgemeine Vorgehensweise zur Betrachtung des Stabilit�tsverhaltens von Einschrittverfahren aufgezeigt werden und an einigen einfachen Verfahren verdeutlicht werden. Dabei wurden die Vor- und Nachteile der impliziten Verfahren gegen�ber den expliziten Verfahren deutlich.
F�r Mehrschrittverfahren konnte eine Methode zur �berpr�fung der Stabilit�t in konkreten Anwendungsf�llen vorgeschlagen werden.\\
Abschlie�end wurden ausgew�hlte, in der Energietechnik zum Einsatz kommende Computerprogramme knapp auf deren verwendete Integrationsverfahren untersucht.

Auf die Konstruktion der verschiedenen Verfahren wurde -- abgesehen von den einfachsten Verfahren -- im Rahmen dieser Arbeit bewusst verzichtet. Diese k�nnen in \cite{gruene} oder entsprechender Literatur nachgelesen werden. \\
Eine weitere Klasse von Verfahren, die sogenannten \textit{Extrapolationsverfahren}, stellen in ihrer Herleitung und Umsetzung eine besondere Form der allgemeinen Runge-Kutta-Verfahren dar und wurde im Zuge dieser Arbeit nicht n�her betrachtet.

 

											% Zusammenfassung (Bezeichnung passend zur Arbeit!)

\appendix																							% Anhang

\chapter{�berblick einiger Verfahren}
\section{Einschrittverfahren}
\begin{longtable}{|lccccc|}\hline
\textbf{Name}&\textbf{Ordnung}&\textbf{Butcher-Schema}&\textbf{Art}&\textbf{Stabilit�t} & Stabilit�tsfunktion/Bereich\\ \hline
Euler &1 & $\begin{array}[c]{c|c}
 0 &  \\ \hline
 & 1
\end{array}$ & explizit & ...\\ \hdashline[1pt/2pt]
%%
Heun & 2 & $\begin{array}[c]{c|cc}
 0   \\
1 & 1 \\ \hline
 & \eh & \eh \end{array}$ & explizit & ... \\\hdashline[1pt/2pt]
%%%
klassisches Runge-Kutta & 4 & $\begin{array}[c]{c|cccc}
		0 &\\ 
		\eh&\eh \\ 
		\eh & 0 & \eh \\
		1&0&0&1 \\ \hline
		& \frac{1}{6}&\frac{2}{6}&\frac{2}{6}&\frac{1}{6}
	\end{array}$ & explizit & ...\\\hdashline[1pt/2pt]
%%%%%%%%%%%
Euler & 1 & $\begin{array}[c]{c|c}
		0 &\\ \hline
		&1
	\end{array}$ & implizit & ...\\\hdashline[1pt/2pt]
Mittelpunktsregel & 2 & $\begin{array}[c]{c|cc}
		0 &\\ 
		1&1 \\ \hline
		& \eh & \eh
	\end{array}$ & implizit & ... \\\hdashline[1pt/2pt]
Trapezregel & 2 &$\begin{array}[c]{c|cccc}
		0 &\\ 
		\eh&\eh \\ 
		\eh & 0 & \eh \\
		1&0&0&1 \\ \hline
		& \frac{1}{6}&\frac{2}{6}&\frac{2}{6}&\frac{1}{6}
	\end{array}$ & implizit & ... \\ \lasthline
\end{longtable}

%%%%%%%%%%%%%%%%%%%%%%%%%%%%%%%%%%%%%%%%%%%%%%
\newpage
\section{Eingebettete Runge-Kutta-Verfahren (Schrittweitensteuerung)}
\begin{longtable}{|lcc|}\hline
\tikzmark[2pt]{1}{\parbox{2cm}{\textbf{Name/\\Ordnung}}}&\textbf{Stufen}&\textbf{Butcher-Schema}\\ \hline
\parbox{2cm}{Fehlberg-RK4(3)} & 5 & $
\begin{array}[c]{c|ccccc}
		0\\
		\eh & \eh \\
		\eh & 0 & \eh \\
		1 & 0 & 0 & 1 \\%
		1 & \frac{1}{6} & \frac{2}{6} & \frac{2}{6} & \frac{1}{6}\\\hline
		& \frac{1}{6} & \frac{2}{6} & \frac{2}{6} & \frac{1}{6} & 0 \\ \hline
		& \frac{1}{6} & \frac{2}{6} & \frac{2}{6} & 0 & \frac{1}{6} 
\end{array}
$ \\ \hdashline[1pt/2pt]
\parbox{2cm}{Dormand-Prince-RK5(4)} & 6 & $
\begin{array}[c]{c|ccccccc}
		0\\
		\frac{1}{5} & \frac{1}{5} \\
		\frac{3}{10} & \frac{3}{40} & \frac{9}{40} \\
		\frac{4}{5} & \frac{44}{45} & -\frac{56}{15} & \frac{32}{9} \\%
		\frac{8}{9} & \frac{19372}{6561} & -\frac{25360}{2187} & \frac{64448}{6561} & -\frac{5103}{18656}\\
		1 & \frac{9017}{3168} & -\frac{355}{33} & \frac{46732}{5247} & \frac{49}{176} & -\frac{5103}{18656}\\
		1 & \frac{35}{384} & 0 & \frac{500}{1113} & \frac{125}{92} & -\frac{2187}{6487} & \frac{11}{84}\\\hline
		& \frac{35}{384} & 0 & \frac{500}{1113} & \frac{125}{92} & -\frac{2187}{6487} & \frac{11}{84}&0 \\ \hline
		& \frac{5179}{57600} & 0 & \frac{7571}{16695} & \frac{393}{640} & -\frac{92097}{339200} & \frac{187}{2100} & \frac{1}{40} 
\end{array}
$ \\ \lasthline
\end{longtable}
%%%%%%%%%%%%%%%%%%%%%%%%%%%%%%%%%%%%%%%%%
\begin{landscape}
\section{Mehrschrittverfahren}
\small
\arraycolsep0pt
\begin{longtable}{|llcccc|}\hline
$k$ &\textbf{Name}&\textbf{$P_a(z)$}&\textbf{$P_b(z)$}&\textbf{Art}&\textbf{Stabilit�t}\\ \hline
$\begin{array}{c}2\end{array}$ & \parbox{1.4cm}{Milne-Simpson} &  $z^2-1$ & $\frac{1}{3}z^2 + \frac{4}{3}z + \frac{1}{3}$ & implizit & ...? (Sinn?) \\[4pt] \hdashline[1pt/2pt]
$\begin{array}[c]{l}
1\\2\\3\\4
\end{array}$&
\parbox{1.4cm}{Adams-Bashforth} & $z^{k}-z^{k-1}$ &$
\begin{array}[c]{l}
	1 \ \text{\footnotesize($\hat{=}$ Euler, explizit)}\\
	(3z-1)/2 \\
	(23z^2-16z+5)/12\\
	(55z^3-59z^2+37z-9)/24
\end{array}$ & explizit & ... \\\hdashline[1pt/2pt]
$\begin{array}[c]{l}
2\\3\\4
\end{array}$&
\parbox{1.4cm}{Adams-Moulton} & $z^{k}-z^{k-1}$ &$
\begin{array}[c]{l}
	(5z^2+8z-1)/12\\
	(9z^3+19z^2-5z+1)/24\\
	(251z^4+646z^3+-246z^2+106z-19)
\end{array}$ & implizit & ... \\[4pt]\hdashline[1pt/2pt]
$\begin{array}[c]{l}
1\\2\\3\\4
\end{array}$&
BDF &%\makebox[2.5cm][l]{$\begin{array}[c]{l}
	$\begin{array}[c]{l}
	z-1 \ \text{\footnotesize($\hat{=}$ Euler, implizit)}\\
	\frac{3}{2}z^2-2z+\eh\\
	\frac{11}{6}z^3-3z^2+\frac{3}{2}z-\frac{1}{3}\\
	\frac{25}{12}z^4-4z^3+3z^2-\frac{4}{3}z+\frac{1}{4}\\
\end{array}$ & $z^k$ & implizit & \parbox{4cm}{unendlich gro�es Stabilit�tsgebiet $\Rightarrow$ gut f�r steife DGL} \\\lasthline
\end{longtable}
\end{landscape}
\chapter{Anhang Teil 2}
\chapter{Anhang Teil 3}























																% Anhangsinhalt

\addchap{Symbol- und Abk�rzungsverzeichnis}
\label{kap:symbole}

Sollten in einer Ausarbeitung viele Abk�rzungen und Formelzeichen auftreten, so
empfiehlt es sich, diese gesondert in einem Kapitel aufzufuhren. Dieses kann auch nach
dem Inhaltsverzeichnis (Abbildungs- und Tabellenverzeichnis) folgen.
\begin{tabbing}
XXXXXX \= XXXX \= XXXXXXXXXXXXXXXXXXXXXXXXXXXXXXXXXXXXXXXXXXXXXXXXX \kill
A											\>																	\> Abk�rzung f�r ... \\
$\cos\varphi$					\>																	\> Leistungsfaktor \\
$U\iz{s}$							\> V																\> Betrag der Statorspannung \\
$\varphi$							\> rad															\> Winkel zwischen Spannung und Strom \\
\end{tabbing}															% Symbol- und Abk�rzungsverzeichnis

\begin{spacing}{1.5}          												% evtl. kleinerer Zeilenabstand im LV
%\nocite{*}                              							% alle Literaturquellen einbinden, sonst werden nur die zitierten
                                            					% Quellen im Literaturverzeichnis angezeigt.
\bibliographystyle{unsrtdin}		         						  % Darstellung numerisch nach Vorkommen im Text
\bibliography{literatur}
\end{spacing}

%\newpage
%\renewcommand{\indexname}{Stichwortverzeichnis}			% Stichwortverzeichnisbezeichnung
%\addcontentsline{toc}{chapter}{Stichwortverzeichnis} % Stichwortverzeichnis ins IV einf?gen
%\begin{spacing}{1.0} 
%\printindex																					% Stichwortverzeichnis anzeigen
%\end{spacing}

\end{document}
%------------------------- Ende des Dokuments -----------------------